%%%%%%%%%%%%%%%%%%%%%%%%%%%%%%%%%%%%%%%%%%%%%%%%%%%%%%%%%%%%%%%%%%%%%%%%
% Plantilla para escribir libros
% Universidad de A Coruña. Facultad de Informática
% Realizado por: Welton Vieira dos Santos
% Modificado: Welton Vieira dos Santos
% Contacto: welton.dosssantos@udc.es
%%%%%%%%%%%%%%%%%%%%%%%%%%%%%%%%%%%%%%%%%%%%%%%%%%%%%%%%%%%%%%%%%%%%%%%%


%%%%%%%%%%%%%%%%%%%%%%%%
% FORMATO DEL DOCUMENTO
%%%%%%%%%%%%%%%%%%%%%%%%
% scrbook es la clase de documento
% Si se desea que no haya página en blanco entre capítulos añadir "openany" en los parámetros de la clase. Sino siempre los capítulos empezarán en página impar.
\documentclass[a4paper,12pt,titlepage]{scrbook}
\KOMAoption{toc}{bib,chapterentryfill} % Opciones del índice
\usepackage{scrhack} % Previene algunos errores
% Paquete de formato para scrbook. Con marcas, linea-separador superior e inferior
\usepackage[automark,headsepline,footsepline]{scrlayer-scrpage}
\clearpairofpagestyles		% Borra los estilos por defecto
%%
% Formato y contenido de la información de cabecera y pie de página
%%
% Información de capítulo en cabecera e interno
\ihead{{\color{gray30}\scshape\small\headmark}}	
% Número de página en cabecera y externo
\ohead{\normalfont\pagemark} 
% Número de página en pie de página y externo. Sólo en páginas sin cabecera
%\ofoot[\normalfont\pagemark]{}
%% 		
% Edición del contenido de las distintas partes de la cabecera
%%
\renewcommand{\chaptermark}[1]{\markboth{#1}{}} % Capítulo (Solo texto)
\renewcommand{\sectionmark}[1]{\markright{\thesection. #1}} % Sección (Número y texto)
\setkomafont{pagenumber}{} % Número de página (Sin nada añadido)

% Añade al índice y numera hasta la profundidad 4.
% 1:section,2:subsection,3:subsubsection,4:paragraph
\setcounter{tocdepth}{2}
\setcounter{secnumdepth}{2}
% Muestra una regla para comprobar el formato de las páginas
%\usepackage[type=upperleft,showframe,marklength=8mm]{fgruler}
% MÁRGENES DE LAS PÁGINAS
\usepackage[
  inner	=	3.0cm, % Margen interior
  outer	=	2.5cm, % Margen exterior
  top	=	2.5cm, % Margen superior
  bottom=	2.5cm, % Margen inferior
  includeheadfoot, % Incluye cabecera y pie de página en los márgenes
]{geometry}
% Valor de interlineado
\renewcommand{\baselinestretch}{1.0} % 1 línea de interlineado
% Para poder generar páginas horizontales
\usepackage{lscape}
% Ancho de la zona para comentarios en el margen. (modificado para todonotes)
\setlength{\marginparwidth}{1.9cm}

%%%%%%%%%%%%%%%%%%%%%%%%
% BIBLIOGRAFÍA
%%%%%%%%%%%%%%%%%%%%%%%%
%\usepackage{apacite} % NORMA APA
\usepackage[numbers,sort]{natbib}
\usepackage{breakcites}

%%%%%%%%%%%%%%%%%%%%%%%%
% DOCUMENTO EN ESPAÑOL
%%%%%%%%%%%%%%%%%%%%%%%%
\usepackage[base]{babel}
\usepackage{polyglossia}
\setdefaultlanguage{spanish}

\addto\captionsspanish{%
	\renewcommand{\listtablename}{Índice de tablas} 
	\renewcommand{\tablename}{Tabla}
	\renewcommand{\lstlistingname}{Código}
	\renewcommand{\lstlistlistingname}{Índice de \lstlistingname s}
	\renewcommand{\glossaryname}{Glosario}
	\renewcommand{\acronymname}{Acrónimos}
	\renewcommand{\bibname}{Bibliografía}%
}

%%%%%%%%%%%%%%%%%%%%%%%% 
% COLORES
%%%%%%%%%%%%%%%%%%%%%%%% 
% Biblioteca de colores
\usepackage{color}
\usepackage[dvipsnames]{xcolor}
% Otros colores definidos por el usuario
\definecolor{gray97}{gray}{.97}
\definecolor{gray75}{gray}{.75}
\definecolor{gray45}{gray}{.45}
\definecolor{gray30}{gray}{.30}
\definecolor{negro}{RGB}{0,0,0}
\definecolor{blanco}{RGB}{255,255,255}
\definecolor{dkgreen}{rgb}{0,.6,0}
\definecolor{dkblue}{rgb}{0,0,.6}
\definecolor{dkyellow}{cmyk}{0,0,.8,.3}
\definecolor{gray}{rgb}{0.5,0.5,0.5}
\definecolor{mauve}{rgb}{0.58,0,0.82}
\definecolor{deepblue}{rgb}{0,0,0.5}
\definecolor{deepred}{rgb}{0.6,0,0}
\definecolor{deepgreen}{rgb}{0,0.5,0}
\definecolor{MyDarkGreen}{rgb}{0.0,0.4,0.0}
\definecolor{bluekeywords}{rgb}{0.13,0.13,1}
\definecolor{greencomments}{rgb}{0,0.5,0}
\definecolor{redstrings}{rgb}{0.9,0,0}

%%%%%%%%%%%%%%%%%%%%%%%%
% TABLAS
%%%%%%%%%%%%%%%%%%%%%%%%
% Paquetes para tablas
\usepackage{longtable,booktabs,array,multirow,tabularx,ragged2e,array}
% Nuevos tipos de columna para tabla, se pueden utilizar como por ejemplo C{3cm} en la definición de columnas de la función tabular
\newcolumntype{L}[1]{>{\raggedright\let\newline\\\arraybackslash\hspace{0pt}}m{#1}}
\newcolumntype{C}[1]{>{\centering\let\newline\\\arraybackslash\hspace{0pt}}m{#1}}
\newcolumntype{R}[1]{>{\raggedleft\let\newline\\\arraybackslash\hspace{0pt}}m{#1}}

%%%%%%%%%%%%%%%%%%%%%%%% 
% GRAFICAS y DIAGRAMAS 
%%%%%%%%%%%%%%%%%%%%%%%% 
% Paquete para todo tipo de gráficas, diagramas, modificación de imágenes, etc
\usepackage{tikz,tikzpagenodes}
\usetikzlibrary{tikzmark,calc,shapes.geometric,arrows,backgrounds,shadings,shapes.arrows,shapes.symbols,shadows,positioning,fit,automata}
\usepackage{pgfplots}
\pgfplotsset{compat=newest} % Compatibilidad
\usepackage{pgfplotstable}
% Estilos para elementos graficos
% Cajas y cajas de texto
\tikzstyle{Caja1} = [green,very thick,rounded corners,fill=white, fill opacity=0.5]
\tikzstyle{Texto1} = [fill=white,thick,shape=circle,draw=black,inner sep=2pt,font=\sffamily,text=black]
\tikzstyle{Texto2} = [fill=white,thick,shape=rectangle,draw=black,inner sep=2pt,font=\sffamily,text=black]
\tikzstyle{Texto3} = [fill=white,thick,shape=circle,draw=black,inner sep=2pt,font=\sffamily,text=black]
% Cuadros de diagrama
\tikzstyle{rectvioleta} = [rectangle, rounded corners, text centered, draw=black, fill=blue!10]
\tikzstyle{rectnaranja} = [rectangle, minimum width=2cm, minimum height=1cm, text centered, draw=black, fill=orange!10]
\tikzstyle{romborosa} = [diamond, aspect=3, minimum width=3cm, minimum height=1cm, text centered, draw=black, fill=red!10]
\tikzstyle{rectverde} = [rectangle, minimum width=2cm, minimum height=1cm, text centered, draw=black, fill=green!10]
\tikzstyle{rectamarillo} = [rectangle, rounded corners, minimum width=2cm, minimum height=1cm, text centered, draw=black, fill=yellow!10]
% Flechas
\tikzstyle{arrow} = [thick,->,>=stealth]

%%%%%%%%%%%%%%%%%%%%%%%% 
% FIGURAS, TABLAS, ETC 
%%%%%%%%%%%%%%%%%%%%%%%% 
\usepackage{subcaption} % Para poder realizar subfiguras
\usepackage{caption} % Para aumentar las opciones de diseño
% Nombres de figuras, tablas, etc, en negrita la numeración, todo con letra small
\captionsetup{labelfont={bf,small},textfont=small}
% Paquete para modificar los espacios arriba y abajo de una figura o tabla
\usepackage{setspace}
% Define el espacio tanto arriba como abajo de las figuras, tablas
\setlength{\intextsep}{5mm}
% Para ajustar tamaños de texto de toda una tabla o grafica
% Uso: {\scalefont{0.8} \begin{...} \end{...} }
\usepackage{scalefnt}
% Redefine las tablas y figuras para eliminar el '.' entre la numeración y el texto
\renewcommand*{\figureformat}{\figurename~\thefigure}
\renewcommand*{\tableformat}{\tablename~\thetable}

%%%%%%%%%%%%%%%%%%%%%%%% 
% TEXTO
%%%%%%%%%%%%%%%%%%%%%%%%
% Paquete para poder modificar las fuente de texto
\usepackage{xltxtra}
% Cualquier tamaño de texto. Uso: {\fontsize{100pt}{120pt}\selectfont tutexto}
\usepackage{anyfontsize}
% Para modificar parametros del texto.
\usepackage{setspace}
% Paquete para posicionar bloques de texto
\usepackage{textpos}
% Paquete para realizar cajas de texto. 
% Uso: \begin{mdframed}[linecolor=red!100!black] tutexto \end{mdframed}
\usepackage{framed,mdframed}
% Para subrayar. Uso: \hlc[tucolor]{tutexto}
\newcommand{\hlc}[2][yellow]{ {\sethlcolor{#1} \hl{#2}} }
% Para que el salto de linea funcione sin tener que usar \\
%\usepackage{parskip}

%%%%%%%%%%%%%%%%%%%%%%%% 
% OTROS
%%%%%%%%%%%%%%%%%%%%%%%%
% Para hacer una pagina horizontal. Uso: \begin{landscape} xxxx \end{lanscape}
\usepackage{lscape} 
% Para incluir paginas PDF. Uso:
% \includepdf[pages={1}]{tuarchivo.pdf}
\usepackage{pdfpages}
% Para introducir url's con formato. Uso: \url{http://www.google.es}
\usepackage{url}
% Amplia muchas funciones graficas de latex
\usepackage{graphicx}
\graphicspath{ {archivos/images/} }
% Paquete que añade el hipervinculo en referencias dentro del documento, indice, etc
% Se define sin bordes alrededor. Uso: \ref{tulabel}
\usepackage[pdfborder={000}]{hyperref}
\usepackage{float}
\usepackage{placeins}
\usepackage{afterpage}
\usepackage{verbatim}
% Paquete para condicionales avanzados
\usepackage{xstring,xifthen}
% Paquete para realizar calculos en el código
\usepackage{calc}
% Para incluir comentrios en el texto. El parámetro 'disable' oculta todas las notas.
% USO: \todo{tutexto}
\usepackage[textsize=tiny,spanish,shadow,textwidth=2cm]{todonotes}
%\reversemarginpar % Descomentar si se quiere todos los comentarios en el mismo lado

%%%%%%%%%%%%%%%%%%%%%%%% 
% GLOSARIOS
%%%%%%%%%%%%%%%%%%%%%%%%
\usepackage[acronym,nonumberlist,toc]{glossaries}
\usepackage{glossary-superragged}
\newglossarystyle{modsuper}{%
  \setglossarystyle{super}%
  \renewcommand{\glsgroupskip}{}
}
\renewcommand{\glsnamefont}[1]{\textbf{#1}}


%%%%%%%%%%%%%%%%%%%%%%%% 
% COMANDOS AÑADIDOS
%%%%%%%%%%%%%%%%%%%%%%%%
% Para mostrar la fecha actual (mes año) con \Hoy
\newcommand{\DIA}{\number\day \space}
\newcommand{\MES}{%
  \ifcase\month% 0
    \or Enero% 1
    \or Febrero% 2
    \or Marzo% 3
    \or Abril% 4
    \or Mayo% 5
    \or Junio% 6
    \or Julio% 7
    \or Agosto% 8
    \or Septiembre% 9
    \or Octubre% 10
    \or Noviembre% 11
    \or Diciembre% 12
  \fi}
\newcommand{\ANYO}{\number\year}
\newcommand{\Hoy}{\DIA de \MES\ de \ANYO}

%%%%%%%%%%%%%%%%%%%%%%%% 
% MATEMÁTICAS
%%%%%%%%%%%%%%%%%%%%%%%%
\RequirePackage{mathtools,amsthm,amsfonts,amssymb,bm,mathrsfs} 
\RequirePackage{upgreek}
% Comando para añadir información de variables a las ecuaciones
% Uso: \begin{condiciones}[donde:] ....... \end{condiciones}
\newenvironment{condiciones}[1][2]
  {%
   #1\tabularx{\textwidth-\widthof{#1}}[t]{
     >{$}l<{$} @{}>{${}}c<{{}$}@{} >{\raggedright\arraybackslash}X
   }%
  }
  {\endtabularx\\[\belowdisplayskip]}

%%%%%
% PARÁMETROS DE FORMATO DE CODIGOS
%%%%%
% Puedes editar los formatos para ajustarlos a tu gusto
%%%%%%%%%%%%%%%%%%%%%%%%%%%%%%%%%%%%%%%%%%%%%%%%%%%%%%%%%%%%%%%%%%%%%
% Plantilla TFG/TFM
% Universidad de Murcia. Facultad de Informática
% Realizado por: José Manuel Requena Plens
% Modificado: Pablo José Rocamora Zamora
% Contacto: pablojoserocamora@gmail.com
%%%%%%%%%%%%%%%%%%%%%%%%%%%%%%%%%%%%%%%%%%%%%%%%%%%%%%%%%%%%%%%%%%%%%


%%%%%%%%%%%%%%%%%%%%%%%% 
% CÓDIGO. CONFIGURACIÓN. En el siguiente bloque están los estilos.
%%%%%%%%%%%%%%%%%%%%%%%%
% Paquete para mostrar código de matlab. En caja y lineas numeradas
\usepackage[framed,numbered]{matlab-prettifier}
% Paquete mostrar código de programación de distintos lenguajes
\usepackage{listings}
\lstset{ inputencoding=utf8,
extendedchars=true,
frame=single, % Caja donde se ubica el código
backgroundcolor=\color{gray97}, % Color del fondo de la caja
rulesepcolor=\color{black},
boxpos=c,
abovecaptionskip=-4pt,
aboveskip=12pt,
belowskip=0pt,
lineskip=0pt,
framerule=0pt,
framextopmargin=4pt,
framexbottommargin=4pt,
framexleftmargin=11pt,
framexrightmargin=0pt,
linewidth=\linewidth,
xleftmargin=\parindent,
framesep=0pt,
rulesep=.4pt,
stringstyle=\ttfamily,
showstringspaces = false,
showspaces = false,
showtabs = false,
columns=fullflexible,
basicstyle=\small\ttfamily,
commentstyle=\color{gray45},
keywordstyle=\bfseries,
tabsize=4,
numbers=left,
numbersep=1pt,
numberstyle=\tiny\ttfamily\color{gray75},
numberfirstline = false,
breaklines=true,
postbreak=\mbox{\textcolor{red}{$\hookrightarrow$}\space}, % Flecha al saltar de linea
prebreak=\mbox{\textcolor{red}{$\hookleftarrow$}\space}, % Flecha al saltar de linea
literate=
  {á}{{\'a}}1 {é}{{\'e}}1 {í}{{\'i}}1 {ó}{{\'o}}1 {ú}{{\'u}}1
  {Á}{{\'A}}1 {É}{{\'E}}1 {Í}{{\'I}}1 {Ó}{{\'O}}1 {Ú}{{\'U}}1
  {à}{{\`a}}1 {è}{{\`e}}1 {ì}{{\`i}}1 {ò}{{\`o}}1 {ù}{{\`u}}1
  {À}{{\`A}}1 {È}{{\'E}}1 {Ì}{{\`I}}1 {Ò}{{\`O}}1 {Ù}{{\`U}}1
  {ä}{{\"a}}1 {ë}{{\"e}}1 {ï}{{\"i}}1 {ö}{{\"o}}1 {ü}{{\"u}}1
  {Ä}{{\"A}}1 {Ë}{{\"E}}1 {Ï}{{\"I}}1 {Ö}{{\"O}}1 {Ü}{{\"U}}1
  {â}{{\^a}}1 {ê}{{\^e}}1 {î}{{\^i}}1 {ô}{{\^o}}1 {û}{{\^u}}1
  {Â}{{\^A}}1 {Ê}{{\^E}}1 {Î}{{\^I}}1 {Ô}{{\^O}}1 {Û}{{\^U}}1
  {œ}{{\oe}}1 {Œ}{{\OE}}1 {æ}{{\ae}}1 {Æ}{{\AE}}1 {ß}{{\ss}}1
  {ű}{{\H{u}}}1 {Ű}{{\H{U}}}1 {ő}{{\H{o}}}1 {Ő}{{\H{O}}}1
  {ç}{{\c c}}1 {Ç}{{\c C}}1 {ø}{{\o}}1 {å}{{\r a}}1 {Å}{{\r A}}1
  {€}{{\euro}}1 {£}{{\pounds}}1 {«}{{\guillemotleft}}1
  {»}{{\guillemotright}}1 {ñ}{{\~n}}1 {Ñ}{{\~N}}1 {¿}{{?`}}1,
  }

% Intenta no dividir los códigos en diferentes paginas si es posible
\lstnewenvironment{listing}[1][]
   {\lstset{#1}\pagebreak[0]}{\pagebreak[0]}

% Formato de títulos de los códigos
\DeclareCaptionFont{white}{\color{white}}
\DeclareCaptionFormat{listing}{\colorbox{gray}{\parbox{\textwidth - 2\fboxsep}{#1#2#3}}}
\captionsetup[lstlisting]{format=listing,labelfont=white,textfont=white,font= scriptsize}


%%%%%%%%%%%%%%%%%%%%%%%% 
% CÓDIGO. ESTILOS. Ajústalos a tu gusto
%%%%%%%%%%%%%%%%%%%%%%%%
\lstdefinestyle{Consola}
	{
	basicstyle=\scriptsize\bf\ttfamily,
	}
   
\lstdefinestyle{C}
	{
	basicstyle=\scriptsize,
	language=C,
	}
\lstdefinestyle{C-color}
	{
  	breaklines=true,
  	language=C,
  	basicstyle=\scriptsize,
  	keywordstyle=\bfseries\color{green!40!black},
  	commentstyle=\itshape\color{purple!40!black},
  	identifierstyle=\color{blue},
  	stringstyle=\color{orange},
    }
\lstdefinestyle{CSharp}
	{
	basicstyle=\scriptsize
	language=[Sharp]C,
	escapeinside={(*@}{@*)},
	keywordstyle=\bfseries,
	}
\lstdefinestyle{CSharp-color}
	{
	basicstyle=\scriptsize
	language=[Sharp]C,
	escapeinside={(*@}{@*)},
	commentstyle=\color{greencomments},
	keywordstyle=\color{bluekeywords}\bfseries,
	stringstyle=\color{redstrings},
	}
\lstdefinestyle{C++}
	{
	basicstyle=\scriptsize,
	language=C++,
 	}
 	
\lstdefinestyle{C++-color}
	{
  	breaklines=true,
  	language=C++,
  	basicstyle=\scriptsize,
  	keywordstyle=\bfseries\color{green!40!black},
  	commentstyle=\itshape\color{purple!40!black},
  	identifierstyle=\color{blue},
  	stringstyle=\color{orange},
    }
    
\lstdefinestyle{PHP}
	{
	basicstyle=\scriptsize,
	language=PHP,
	}
	
\lstdefinestyle{PHP-color}
	{
	basicstyle=\scriptsize,
	language=PHP,
	keywordstyle    = \color{dkblue},
  	stringstyle     = \color{red},
  	identifierstyle = \color{dkgreen},
  	commentstyle    = \color{gray},
  	emph            =[1]{php},
  	emphstyle       =[1]\color{black},
  	emph            =[2]{if,and,or,else},
  	emphstyle       =[2]\color{dkyellow}
  }
  
\lstdefinestyle{Matlab}
	{
	basicstyle=\scriptsize,
	language=Matlab,
	numberstyle=\tiny\ttfamily\color{gray75},
	}
	
\lstdefinestyle{Matlab-color}
	{
	style = Matlab-editor,
	basicstyle=\scriptsize,
	numberstyle=\tiny\ttfamily\color{gray75},
	}
	
\lstdefinestyle{Latex}
	{
	language=[LaTeX]{Tex},
    basicstyle=\scriptsize,
    literate={\$}{{{\bfseries\$}}}1,
    alsoletter={\\,*,\&},
    emph =[1]{\\begin,\\end,\\caption,\\label,\\centering,\\FloatBarrier,
              \\lstinputlisting,\\scalefont,\\addplot,\\input,
              \\legend,\\item,\\subitem,\\includegraphics,\\textwidth,
              \\section,\\subsection,\\subsubsection,\\paragraph,
              \\cite,\\citet,\\citep,\\gls,\\bibliographystyle,\\url,
              \\citet*,\\citep*,\\todo,\\missingfigure,\\footnote},
  	emphstyle =[1]\bfseries,
  	emph = [2]{equation,subequations,eqnarray,figure,subfigure,
  			   condiciones,flalign,tikzpicture,axis,lstlisting,
  			   itemize,description
  			   },
  	emphstyle =[2]\bfseries,
    numbers=none,
	}
	
\lstdefinestyle{Latex-color}
	{
	language=[LaTeX]{Tex},
    basicstyle=\scriptsize,
    commentstyle=\color{dkgreen},
    identifierstyle=\color{black},
    literate={\$}{{{\bfseries\color{Dandelion}\$}}}1, % Colorea el simbolo dollar
    alsoletter={\\,*,\&},
    emph =[1]{\\begin,\\end,\\caption,\\label,\\centering,\\FloatBarrier,
              \\lstinputlisting,\\scalefont,\\addplot,\\input,
              \\legend,\\item,\\subitem,\\includegraphics,\\textwidth,
              \\section,\\subsection,\\subsubsection,\\paragraph,
              \\cite,\\citet,\\citep,\\gls,\\bibliographystyle,\\url,
              \\citet*,\\citep*,\\todo,\\missingfigure,\\footnote},
  	emphstyle =[1]\bfseries\color{RoyalBlue},
  	emph = [2]{equation,subequations,eqnarray,figure,subfigure,
  			   condiciones,flalign,tikzpicture,axis,lstlisting,
  			   itemize,description
  			   },
  	emphstyle =[2]\bfseries,
    numbers=none,
	}
\lstdefinestyle{Java}
{
	basicstyle=\scriptsize,
	language=Java,
}

\lstdefinestyle{Java-color}
{
	basicstyle=\scriptsize,
	language=Java,
  	keywordstyle=\color{blue},
  	commentstyle=\color{dkgreen},
  	stringstyle=\color{mauve},
}
\lstdefinestyle{Python}
{
	language=Python,
	basicstyle=\scriptsize,
	otherkeywords={self},  
	keywordstyle=\bfseries,     
	emphstyle=\bfseries,    
	emph={MyClass,__init__},         
}

\lstdefinestyle{Python-color}
{
	language=Python,
	basicstyle=\scriptsize,
	otherkeywords={self},          
	keywordstyle=\bfseries\color{deepblue},
	emph={MyClass,__init__},         
	emphstyle=\bfseries\color{deepred},    
	stringstyle=\color{deepgreen},
}
\lstdefinestyle{R}
{
	language=R,                     
  	basicstyle=\scriptsize,
  	keywordstyle=\bfseries, 
}
\lstdefinestyle{R-color}
{
	language=R,                     
  	basicstyle=\scriptsize,
  	keywordstyle=\bfseries\color{RoyalBlue}, 
  	commentstyle=\color{YellowGreen},
  	stringstyle=\color{ForestGreen}  
}


%%%%%
% DEFINICION DE CONCEPTOS
%%%%
% Uso ejemplo: \begin{ejemplo} tucontenido \end{ejemplo} 
\newtheorem{teorema}{Teorema}[chapter]
\newtheorem{ejemplo}{Ejemplo}[chapter]
\newtheorem{definicion}{Definición}[chapter]

