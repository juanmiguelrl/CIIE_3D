% !TeX document-id = {c6282027-a510-40f1-b05e-2f558a6cd3f1}
% !TeX program = xelatex
% !TeX TXS-program:compile = txs:///xelatex/[--shell-escape]
%%%%%%%%%%%%%%%%%%%%%%%%%%%%%%%%%%%%%%%%%%%%%%%%%%%%%%%%%%%%%%%%%%%%%%%%
% Plantilla para escribir libros
% Universidad de A Coruña. Facultad de Informática
% Realizado por: Welton Vieira dos Santos
% Modificado: Welton Vieira dos Santos
% Contacto: welton.dosssantos@udc.es
%%%%%%%%%%%%%%%%%%%%%%%%%%%%%%%%%%%%%%%%%%%%%%%%%%%%%%%%%%%%%%%%%%%%%%%%


% Archivo .TEX que incluye todas las configuraciones del documento y los paquetes. Añade todo aquello que necesites utilizar en el documento en este archivo.
% En él se encuentra la configuración de los márgenes, establecidos según las directrices de estilo de la EPS.
\input{include/configuracioninicial}

%%%%%%%%%%%%%%%%%%%%%%%%%%%%%%%%%%%%%%%%%%%%%%%%%%%%%%%%%%%%%%%%%%%%%%
% INFORMACIÓN DEL TFG
% Comentar lo que NO se desee añadir y sustituir con la información correcta.
%%%%%%%%%%%%%%%%%%%%%%%%%%%%%%%%%%%%%%%%%%%%%%%%%%%%%%%%%%%%%%%%%%%%%%
% Título y subtítulo
\newcommand{\titulo}{Entornos Inmersivos, Interactivos y de Entretenimiento}
\newcommand{\subtitulo}{Memoria Artística}
% Datos del autor
\newcommand{\miNombre}{Elena Goyanes González\\Rocío Rego Sierra\\Martiño Moure Vila\\Welton Vieira dos Santos \\Juan Miguel Regal Llamas}
\newcommand{\miDNI}{49205764-R}
\newcommand{\miEmail}{welton.dossantos@udc.es}
%% Datos del tutor/es
\newcommand{\miTutor}{Julián Alfonso Dorado De La Calle (julian.dorado@udc.es)}
\newcommand{\miTutorB}{Enrique Fernandéz Blanco (enrique.fernandez@udc.es)}
\newcommand{\departamentoTutor}{Ciencias de la Computación y Tecnologías de la Información}
\newcommand{\departamentoTutorB}{Ciencias de la Computación y Tecnologías de la Información}
%% Datos de la facultada y universidad
\newcommand{\miFacultad}{Facultad de Informática}
\newcommand{\miFacultadCorto}{FIC}
\newcommand{\miUniversidad}{\protect{Universidad de La Coruña}}
\newcommand{\miUbicacion}{La Coruña}

\newcommand{\firma}{include/firma}

% Configuración automática según el identificador elegido
% Grados
\definecolor{informatica}{RGB}{121,11,21}	% Informatica
% Colores generales
\definecolor{negro}{RGB}{0,0,0}
\definecolor{blanco}{RGB}{255,255,255}

% Logotipos comunes de todas las titulaciones
\newcommand{\logoFacultadPortada}{include/logos-titulaciones/Fic}
\newcommand{\logoGradoPortada}{include/logos-titulaciones/Simbolo-negativo-sin-fondo} 
\newcommand{\logoUniversidadPortada}{include/logos-universidad/EscudoUDC}
\newcommand{\logoFacultadPortadaBaja}{include/logos-titulaciones/Logo_fic}
% Logos
\newcommand{\logoGrado}{include/logos-titulaciones/Logo_fic}
\newcommand{\logoDepartamento}{include/logos-titulaciones/Fic}
\newcommand{\logoUniversidad}{include/logos-universidad/LogoUDC}
% Texto
\newcommand{\miGrado}{Grado en Ingeniería Informática}
\newcommand{\tipotrabajo}{Grupo 2C - ISS Manticore}
% Color
\newcommand{\colorgrado}{informatica}
\newcommand{\colortexto}{blanco}

% Información añadida a las propiedades del archivo PDF.
\hypersetup{
	pdfauthor = {\miNombre~(\miEmail)},
	pdftitle = {\titulo},
}

%%
% Archivo de acrónimos
%%
\makeglossaries % Genera la base de datos de acrónimos
\input{anexos/_acronimos.tex} % Archivo que contiene los acrónimos


%%%%%%%%%%%%%%%%%%%%%%%% 
% INICIO DEL DOCUMENTO
% A partir de aquí debes empezar a realizar tu TFG/TFM
%%%%%%%%%%%%%%%%%%%%%%%%
\begin{document}
	
	% Números romanos hasta el mainmatter.
	\frontmatter
	
	% PORTADA
	\input{include/portada/portada_color} % Portada Color
	%%%%%%%%%%%%%%%%%%%%%%%%%%%%%%%%%%%%%%%%%%%%%%%%%%%%%%%%%%%%%%%%%%%%%%%%
% Plantilla para escribir libros
% Universidad de A Coruña. Facultad de Informática
% Realizado por: Welton Vieira dos Santos
% Modificado: Welton Vieira dos Santos
% Contacto: welton.dosssantos@udc.es
%%%%%%%%%%%%%%%%%%%%%%%%%%%%%%%%%%%%%%%%%%%%%%%%%%%%%%%%%%%%%%%%%%%%%%%%


\begin{titlepage}

	% Márgenes de esta pagina modificados
	\newgeometry{ignoreall,top=2cm,ignoreall}
	
	% Offset horizontal para toda la portada
	\setlength{\centeroffset}{-0.5\oddsidemargin}
	\addtolength{\centeroffset}{0.5\evensidemargin}
	\thispagestyle{empty}
	
	% Titulo y subtitulo
	\noindent\hspace*{\centeroffset}\begin{minipage}{\textwidth}
	\centering
	\begin{spacing}{1.5}{\huge\bfseries \titulo}\end{spacing}
		\noindent\rule[-1ex]{\textwidth}{3pt}\\[3.5ex] % Linea
		{\large\bfseries \subtitulo\\[4cm]}
	\end{minipage}
	
	% Relleno hasta la zona central
	\vspace{2.5cm}
	
	% Zona central. Autor y Tutores
	\noindent\hspace*{\centeroffset}
	\begin{minipage}{\textwidth}
		\centering
		
		\textbf{Autor}\\ {\miNombre}\\[2.5ex]
		\textbf{Tutor/es}\\
		{\normalsize \miTutor\\
		\ifx\departamentoTutor\undefined \else \small\textit \departamentoTutor\\ \fi
		\ifx\miTutorB\undefined \else \normalsize \miTutorB\\ \fi
		\ifx\departamentoTutorB\undefined \else\small\textit \departamentoTutorB\\[2cm] \fi
		}
	\end{minipage}
	
	% Relleno hasta la zona de abajo
	\vspace*{\fill}
	
	% Zona de abajo
	\noindent\hspace*{\centeroffset}
	\begin{minipage}{\textwidth}
		\centering
		\noindent\hspace*{\centeroffset}
		\begin{center}
			{\includegraphics[width=3cm]{\logoGrado}}\\
			{\raggedleft\miGrado}
		\end{center}
		\vspace*{2em}
		\centering
		\noindent\hspace*{\centeroffset}
		\begin{minipage}[l]{6cm}
			\includegraphics[width=2cm]{\logoDepartamento}
		\end{minipage}
		\begin{minipage}[r]{6cm}
			\includegraphics[width=6cm]{\logoUniversidad}
		\end{minipage}
		\\[1cm]
		A Coruña, \Hoy
	\end{minipage}


\end{titlepage}

% A partir de aquí aplica los márgenes establecidos en configuracioninicial.tex
\restoregeometry


 % Portada B/N
	
	%%%%% PREAMBULO
	% Incluye el .tex que contiene el preámbulo, agradecimientos y dedicatorias.
	%\setstretch{1.5} % 1.5 línea de interlineado antes esta a 1 para la portada
	%\input{preambulos/preliminaresconagradecimientos}
	%%%%%%%%%%%%%%%%%%%%%%%%%%%%%%%%%%%%%%%%%%%%%%%%%%%%%%%%%%%%%%%%%%%%%%%%%
% Plantilla TFG/TFM
% Universidad de Murcia. Facultad de Informática
% Realizado por: José Manuel Requena Plens
% Modificado: Pablo José Rocamora Zamora
% Contacto: pablojoserocamora@gmail.com
%%%%%%%%%%%%%%%%%%%%%%%%%%%%%%%%%%%%%%%%%%%%%%%%%%%%%%%%%%%%%%%%%%%%%%%%


% Modificamos los margenes para esta pagina
\newgeometry{left=3.0cm,right=2.5cm,top=2.5cm,bottom=2cm}


\chapter*{\centering Declaración firmada sobre originalidad del trabajo}

D./Dña. \textbf{\miNombre}, con DNI \textbf{\miDNI}, estudiante de la
titulación de \textbf{\miGrado} de la Universidad de Murcia y autor del
TF titulado ``\textbf{\titulo}''.

\vspace{1cm}

De acuerdo con el Reglamento por el que se regulan los Trabajos Fin de
Grado y de Fin de Máster en la Universidad de Murcia (aprobado C. de
Gob. 30-04-2015, modificado 22-04-2016 y 28-09-2018), así como la
normativa interna para la oferta, asignación, elaboración y defensa
delos Trabajos Fin de Grado y Fin de Máster de las titulaciones
impartidas en la Facultad de Informática de la Universidad de Murcia
(aprobada en Junta de Facultad 27-11-2015)

\vspace{1cm}

DECLARO:

\vspace{0.5cm}

Que el Trabajo Fin de Grado presentado para su evaluación es original y
de elaboración personal. Todas las fuentes utilizadas han sido
debidamente citadas. Así mismo, declara que no incumple ningún contrato
de confidencialidad, ni viola ningún derecho de propiedad intelectual e
industrial

\begin{center}\miUbicacion, a \Hoy\end{center}

%\begin{center}\includegraphics[width=3.2cm]{\firma}\end{center}
\missingfigure{Añadir firma}

\begin{center}Fdo.: \miNombre\\
Autor del TF\end{center}


\restoregeometry

	%%%%%%%%%%%%%%%%%%%%%%%%%%%%%%%%%%%%%%%%%%%%%%%%%%%%%%%%%%%%%%%%%%%%%%%%%
% Plantilla para escribir libros
% Universidad de A Coruña. Facultad de Informática
% Realizado por: Welton Vieira dos Santos
% Modificado: Welton Vieira dos Santos
% Contacto: welton.dosssantos@udc.es
%%%%%%%%%%%%%%%%%%%%%%%%%%%%%%%%%%%%%%%%%%%%%%%%%%%%%%%%%%%%%%%%%%%%%%%%


\chapter*{Resumen}



\todo[inline]{}
	
Este resumen se ha hecho para facilitar el estudio de la asignatura y dar una buena base para hacer el examen de teoria y esta divido en nueve temas:

\begin{itemize}
    \item Tema 1: Introducción
    \item Tema 2: La Imagen Digital
    \item Ejercicios del tema 1 y 2
    \item Tema 3: Preprocesado de Imágenes
    \item Tema 4: Filtros - Procesamiento local
    \item Ejercicios del tema 3 y 4
    \item Tema 5: Puntos Característicos
    \item Tema 6: Segmentación Basada en Regiones
    \item Tema 7: Segmentación Basada en Bordes
    \item Tema 8: Evaluación de la Segmentación
    \item Tema 9: Reconocimiento  de Patrones
\end{itemize}    





	%%%%%%%%%%%%%%%%%%%%%%%%%%%%%%%%%%%%%%%%%%%%%%%%%%%%%%%%%%%%%%%%%%%%%%%%%
% Plantilla TFG/TFM
% Universidad de Murcia. Facultad de Informática
% Realizado por: José Manuel Requena Plens
% Modificado: Pablo José Rocamora Zamora
% Contacto: pablojoserocamora@gmail.com
%%%%%%%%%%%%%%%%%%%%%%%%%%%%%%%%%%%%%%%%%%%%%%%%%%%%%%%%%%%%%%%%%%%%%%%%


\chapter*{Extended Abstract}


Resumen extendido en inglés bajo el título “Extended Abstract”. Este apartado se situará tras el apartado “Resumen” y tendrá una extensión mínima de 2000 palabras.

\todo{Falta por hacer}

Lorem ipsum dolor sit amet, consectetur adipiscing elit. Mauris at ligula dolor. Cras sodales porttitor tellus, vel eleifend lectus porta vel. Cras dignissim nec ex at venenatis. Aliquam erat volutpat. Fusce aliquet bibendum mauris id convallis. Integer tempus maximus vehicula. Morbi eu lorem a nibh faucibus viverra. Proin tellus tellus, euismod in est id, ultricies blandit urna. Phasellus tincidunt nec massa a efficitur. Nulla non purus purus.

Sed vestibulum placerat malesuada. Quisque in libero nulla. Suspendisse rhoncus vitae ante ut pretium. Maecenas efficitur nisl non luctus eleifend. Nam convallis lobortis elit in pulvinar. Duis pellentesque dui ac iaculis bibendum. Etiam maximus viverra velit eu sollicitudin. Morbi pharetra, mi vel commodo tincidunt, urna dolor facilisis orci, et tristique odio nisl in ex.

Vestibulum ante ipsum primis in faucibus orci luctus et ultrices posuere cubilia Curae; Suspendisse vehicula non nibh vitae fringilla. Nulla fermentum dolor at rutrum pharetra. Aenean vel nulla lacus. Duis pharetra et metus et tempor. Nam eu lectus rutrum, mollis nulla at, consectetur orci. Nam auctor dictum iaculis. Integer quis urna nisi. Nunc at erat nibh. Sed a auctor nulla, quis aliquet odio. Vivamus enim mauris, ultricies auctor tellus in, aliquet placerat magna.

Etiam sagittis, purus in tincidunt commodo, erat ligula egestas massa, eu bibendum est velit interdum massa. Phasellus sapien purus, blandit non sem a, tristique cursus sapien. In elit velit, volutpat eu lorem vitae, gravida consequat erat. Sed eu lacinia quam. Duis sit amet urna ac nulla sollicitudin elementum. Nam blandit quam ac elit auctor tincidunt. Etiam dui nunc, blandit vitae purus in, commodo tincidunt ante. Vivamus viverra dui metus, vel dictum nisi congue a. Nam a dapibus mauris, eget feugiat enim. In hac habitasse platea dictumst. Etiam at mollis tortor. Aliquam erat volutpat. Aliquam molestie scelerisque tortor vel suscipit.

Cras sodales justo vitae ex egestas, sed ultricies metus suscipit. Donec sed est eget ex scelerisque pharetra. Donec scelerisque tempor mi eu malesuada. Interdum et malesuada fames ac ante ipsum primis in faucibus. Curabitur nisl ante, hendrerit sit amet tortor in, volutpat tincidunt lacus. Aenean rutrum volutpat velit et lobortis. Aliquam lorem magna, iaculis vel pellentesque ac, bibendum nec nibh. Proin dictum libero vel ante viverra, placerat pharetra augue tempor. Nunc vel erat sed felis lobortis tristique eu a odio. Morbi vel lorem nec eros gravida aliquet eget in risus. Aliquam in justo volutpat, tempor dui et, convallis ex. Nulla id condimentum ipsum, nec hendrerit est. Praesent semper arcu sit amet tincidunt ultrices. Aenean quis cursus leo, ut pulvinar turpis.

Nulla vel ex sed sem consequat pretium. Duis lobortis rutrum mi, non efficitur tortor porta vitae. In hac habitasse platea dictumst. Aenean convallis felis a ante faucibus, a consectetur magna tempus. Praesent volutpat cursus elit, dictum mollis sapien ultrices sit amet. Nunc pharetra vestibulum mi eu iaculis. Proin commodo dui nisl, sit amet faucibus augue dignissim fermentum. Pellentesque ultricies elit sit amet eros interdum feugiat. Fusce urna nulla, iaculis et velit sed, euismod ultrices lorem. Nunc mi tortor, porttitor quis rutrum et, blandit vel nunc. Cras et enim faucibus, placerat justo vel, semper diam.

Suspendisse rutrum, nibh vel iaculis facilisis, diam quam sollicitudin neque, et fringilla nunc nisi ut diam. Duis porttitor arcu nulla, vitae viverra diam laoreet et. Nullam in ligula vel eros tincidunt egestas. Donec id magna sed risus pretium commodo. Fusce vulputate lectus eget ipsum porta accumsan. Nunc consequat, nisi eleifend auctor eleifend, ipsum metus semper dolor, eget scelerisque est libero ut turpis. Mauris facilisis a odio eget sodales. Quisque eget enim placerat, ornare ex vel, aliquam mi. Sed in lacinia leo. Vestibulum a egestas massa, faucibus dapibus enim. Pellentesque aliquet, tortor id dignissim tempus, orci quam mollis sem, sit amet dignissim odio urna in tellus. Cras non libero non quam tincidunt lobortis auctor in nulla.

Proin eu consectetur felis. Vivamus consequat neque ac diam viverra, sed venenatis risus mattis. Pellentesque cursus enim iaculis metus convallis dapibus. Suspendisse potenti. Praesent non metus porta tellus vehicula elementum. Curabitur fermentum erat eu consequat aliquet. Curabitur eget massa eu tortor vestibulum vulputate sit amet in nunc. Aenean fermentum sodales mauris at sodales. Nullam aliquet eros turpis, a posuere diam dignissim vel. Curabitur velit massa, sollicitudin et velit sed, condimentum dictum enim. Donec et dolor augue. Quisque vulputate scelerisque nunc.

Class aptent taciti sociosqu ad litora torquent per conubia nostra, per inceptos himenaeos. Aenean varius rhoncus quam, ut euismod magna pretium et. Aenean ultrices, enim id fermentum consequat, odio leo consectetur orci, et gravida nisi nisi nec felis. Pellentesque ipsum elit, sollicitudin ac metus non, aliquam sollicitudin odio. Phasellus ipsum ante, laoreet accumsan metus nec, ultricies faucibus nulla. Mauris a elementum ligula, vel cursus lectus. Sed id aliquet turpis, egestas hendrerit velit. Donec aliquet id dui a elementum. Duis facilisis fermentum sodales. Morbi eget placerat risus. Nulla facilisi. Etiam vestibulum massa et eros pulvinar gravida. Maecenas pellentesque ex eget enim congue sagittis.

Pellentesque sed dui laoreet, pretium massa sit amet, placerat quam. Nulla et enim in nulla ultrices rutrum id finibus ex. Integer tincidunt blandit nunc et tincidunt. Duis gravida hendrerit neque ut tincidunt. Interdum et malesuada fames ac ante ipsum primis in faucibus. Praesent dictum malesuada blandit. Praesent eget fermentum nisl. Nam in libero massa. Vivamus volutpat varius ante in fringilla. Sed eu mi risus. Donec cursus arcu quis quam congue tempus. Maecenas turpis nulla, rhoncus et mi ut, accumsan vestibulum massa. Suspendisse aliquet ullamcorper metus ut dictum.

Nullam blandit quis tellus nec ullamcorper. Phasellus dapibus mauris sit amet lorem cursus, vitae imperdiet risus accumsan. Suspendisse mollis sollicitudin metus nec facilisis. Sed turpis nisl, posuere non purus ut, sagittis sodales erat. Nam sagittis sagittis quam quis ornare. Donec rhoncus turpis porta lacus congue dictum. Aliquam suscipit consectetur lobortis. Aliquam fringilla risus ut hendrerit aliquet. Nulla faucibus, ante vel congue volutpat, ex est facilisis odio, eget semper nibh elit in ipsum. Duis mollis, nulla a sollicitudin rutrum, ex diam efficitur massa, vitae varius est velit eu justo. Nunc sed neque nec libero semper cursus non vitae diam. Curabitur nec iaculis sem. Quisque ut dignissim urna.

Vivamus interdum vel turpis ac rutrum. Integer a metus ut odio consectetur facilisis et sed nulla. In et est ut mauris suscipit commodo non sit amet ex. Nulla facilisi. Sed id semper lectus. Fusce egestas dolor a scelerisque consequat. Nunc maximus mi eget tempor feugiat. Praesent mauris metus, congue ac interdum luctus, fermentum id ante. Nam porttitor dignissim leo et ultrices. Proin imperdiet lectus nisl, vel tristique lacus vestibulum ac. Vivamus urna turpis, sollicitudin ac dolor sit amet, tincidunt consequat lacus. Sed faucibus a ligula nec tincidunt.

Sed aliquet tincidunt nisl, nec pharetra elit molestie eget. Quisque justo ex, aliquet ut suscipit quis, dapibus sit amet diam. Mauris mollis egestas fringilla. Mauris ornare lorem id tellus consectetur malesuada. Suspendisse a orci nibh. Nullam placerat arcu in arcu cursus tempor. Aliquam non felis dolor. Sed luctus nisl at vehicula bibendum. Integer tellus ante, egestas eu sollicitudin ut, tristique vitae eros. Ut dolor diam, rhoncus vitae ante quis, tincidunt pretium dui. Morbi ipsum odio, viverra et tempor vel, dictum a justo. Cras aliquam quam sed justo bibendum feugiat. Quisque in nisl ut neque blandit accumsan imperdiet non leo.

Suspendisse laoreet nunc id ipsum gravida, sit amet euismod risus maximus. Cras pharetra ipsum et odio tempor tristique. Integer tempor erat eu malesuada imperdiet. Suspendisse scelerisque justo sit amet enim tempus dapibus. Etiam pellentesque maximus velit nec ullamcorper. Vestibulum quam nisi, accumsan porttitor magna sed, fermentum placerat enim. Ut iaculis egestas bibendum. Maecenas dapibus congue lectus laoreet sagittis. Fusce sit amet semper nunc. Donec varius mi tellus, quis cursus nulla aliquet id. Vivamus metus enim, fringilla quis sem sed, egestas vehicula purus. Fusce eget diam et purus tristique tincidunt id eget mauris. Aliquam laoreet consequat nibh ut varius. Morbi volutpat arcu sed pellentesque tempus. Curabitur semper, ipsum in lobortis rhoncus, nibh lacus posuere tellus, eget imperdiet nulla sem vel mauris.

Sed sed risus justo. Donec rutrum sagittis porttitor. Aenean aliquam rhoncus ligula. Proin ex dolor, ultricies in felis ut, porta eleifend justo. Aenean purus augue, viverra a massa nec, aliquam tempor magna. Integer a convallis lacus. Cras lacus sem, pellentesque eu ligula ut, pretium porttitor nisl. Sed volutpat lorem eget arcu gravida viverra. Nulla a scelerisque massa. Etiam ut massa dignissim, efficitur ligula quis, mollis dui. Nunc facilisis neque at dui fringilla, ac volutpat sapien ultrices. Vestibulum tristique mollis luctus. Vivamus eget dignissim odio, sit amet molestie ipsum. In ut est est. Nam a accumsan tellus. Morbi maximus malesuada molestie.

In quis vestibulum nibh. Vestibulum ante ipsum primis in faucibus orci luctus et ultrices posuere cubilia Curae; Praesent dolor purus, egestas at sapien vel, iaculis consectetur nulla. Vivamus non dictum urna, ac maximus tellus. Donec orci ipsum, volutpat ac nibh eget, hendrerit rutrum massa. Sed nulla quam, maximus a pharetra eget, sodales at ipsum. Nulla non arcu neque. Nam dapibus porttitor lacus in hendrerit. Donec at eros nunc. Suspendisse ut lorem ligula. Phasellus sapien mauris, tincidunt et dolor quis, euismod tincidunt leo. Nam aliquam augue purus, nec elementum est congue in. Morbi metus ex, fermentum vitae lorem sed, dignissim porttitor nibh. Ut gravida odio dui, eget iaculis est commodo vel. Aliquam fringilla elit ipsum, ut ornare lectus rhoncus a. Nulla luctus diam vitae lectus pretium varius.

Morbi hendrerit vestibulum metus eu rhoncus. Donec euismod diam ex, vitae ultrices sem consectetur et. Nam felis diam, efficitur in mauris vitae, ullamcorper scelerisque est. In fermentum velit dui, eget tincidunt risus elementum sed. Praesent eleifend imperdiet tellus. Vivamus eu dapibus urna. In egestas blandit metus id eleifend. Vestibulum massa dui, dictum sit amet ante vel, dictum auctor libero. Quisque laoreet, mi iaculis feugiat commodo, urna lectus pretium urna, quis ultrices ipsum erat eu turpis. Fusce aliquam libero in nisi viverra, vel ultrices arcu porttitor. Aliquam erat volutpat. Sed sagittis tellus nec libero ullamcorper tempor. Nullam accumsan nibh quis risus fermentum eleifend.

Curabitur leo elit, hendrerit semper purus sed, ullamcorper iaculis quam. Vivamus ac velit tristique, dapibus orci at, feugiat odio. Proin vel libero est. In hac habitasse platea dictumst. Suspendisse potenti. Fusce vestibulum elit in est accumsan pharetra. Maecenas ac fringilla lectus, at condimentum ipsum. Curabitur aliquet arcu ex. Nam fringilla interdum nibh ut vulputate.

Vivamus ut est nec elit volutpat ultrices. Proin ultrices viverra felis ut scelerisque. Suspendisse ut justo urna. Duis neque urna, congue eu velit vel, molestie rutrum enim. Ut pulvinar in justo vel dapibus. Duis at imperdiet augue, sed bibendum felis. Pellentesque venenatis nisl blandit arcu suscipit molestie. Curabitur tellus mauris, molestie tincidunt quam a, mattis porta tortor. Donec faucibus mauris id nisl elementum, at ornare metus vehicula. Phasellus vel ultrices quam. Praesent fermentum, dolor id gravida sagittis, nisl dui euismod dolor, quis tristique ex urna condimentum tellus. Etiam iaculis metus magna, ac porta velit rhoncus ut.

Sed tempus risus id elit fringilla suscipit. Suspendisse porta justo ut augue hendrerit faucibus. Donec fringilla leo vitae velit mattis tincidunt. Nam vestibulum erat tortor, vitae scelerisque sapien sodales sed. Interdum et malesuada fames ac ante ipsum primis in faucibus. Suspendisse fermentum orci nisi, id dapibus lectus ultrices eget. Nunc id magna lacinia, convallis augue sit amet, tempor ipsum. Donec rutrum maximus velit vel tincidunt. Donec blandit purus quis leo bibendum, nec ornare velit luctus. Cras molestie nunc ut est rutrum, at luctus justo porta. Quisque tristique nisi ac augue rutrum, id posuere quam vulputate. Mauris ultricies facilisis nisl sed dignissim. Nulla eleifend velit sed dui bibendum, ut consectetur libero posuere. Proin sit amet tempor metus. Cras dignissim leo massa, non ultrices sem posuere a.

Sed vitae turpis nec dolor elementum sagittis vel quis eros. Nullam finibus maximus enim eu maximus. Aenean ac egestas risus. Aliquam rhoncus, elit quis vestibulum hendrerit, risus est feugiat ipsum, vitae egestas ipsum lorem vitae lacus. Maecenas vitae odio sit amet velit porta sodales. Lorem ipsum dolor sit amet, consectetur adipiscing elit. Pellentesque quis aliquam urna.

Sed interdum sed orci eu tempor. Fusce imperdiet maximus euismod. Fusce ac porttitor dolor, in placerat risus. In laoreet metus sed pulvinar cursus. Duis sollicitudin ex sit amet enim consequat tempus. Donec ipsum quam, tristique ut elit ut, laoreet aliquam augue. Suspendisse rhoncus massa eu ex vehicula sollicitudin vitae a ipsum. Quisque finibus quam vel nunc feugiat, sit amet posuere justo suscipit. Ut suscipit maximus ante, dictum tempus neque sodales vel. Aliquam erat volutpat.

Aliquam elementum rhoncus quam vel tempor. Vivamus eu erat eget tortor tincidunt congue nec semper lacus. Curabitur eu orci eu tellus rutrum euismod. Suspendisse maximus mi a magna semper tincidunt. Aliquam vitae odio id enim accumsan sodales. Suspendisse viverra libero vel lorem bibendum, nec varius sapien consequat. Donec ullamcorper pellentesque tellus, eu laoreet massa blandit sit amet.

Praesent eget purus at risus accumsan aliquet. In interdum sem non efficitur hendrerit. Nunc molestie convallis vulputate. Maecenas faucibus nec leo at tristique. Interdum et malesuada fames ac ante ipsum primis in faucibus. Lorem ipsum dolor sit amet, consectetur adipiscing elit. Aliquam tincidunt faucibus nisl non ullamcorper.

Praesent quis enim quis mauris dapibus pharetra. Nam iaculis eros sit amet eleifend euismod. Integer bibendum leo in facilisis egestas. Proin cursus consequat velit, ut gravida turpis elementum non. Mauris semper pellentesque elementum. Pellentesque consequat, massa id volutpat cursus, dolor ligula feugiat ligula, in faucibus purus felis ut sem. Nunc vitae ultricies magna. Sed sit amet lectus velit. Duis scelerisque, quam hendrerit sollicitudin placerat, leo turpis sodales est, ut interdum leo tortor at mi. Nunc sit amet orci erat. Mauris leo justo, rutrum eu tortor a, tincidunt elementum sapien. Pellentesque euismod sapien eget dui eleifend, at pulvinar metus pretium. Sed dignissim erat sed quam iaculis mollis eget et augue. Etiam a arcu consectetur, tempus dui ac, finibus mauris.

Mauris placerat nunc aliquam dapibus convallis. Nulla hendrerit risus quis nisi sodales, ut euismod sapien faucibus. Vestibulum sollicitudin id quam id malesuada. Vivamus vestibulum, orci lacinia rutrum semper, ex urna semper justo, at faucibus mi felis ut augue. Nulla facilisi. Sed at. 
 

	
	%\setstretch{1.0} % 1.0 línea de interlineado para los indices
	% Incluye después del archivo anterior el indice y lista de figuras, tablas y códigos.

	\tableofcontents     	% Índice
	\listoffigures	     	% Índice de figuras
	%\listoftables	     	% Índice de tablas
	%\lstlistoflistings 	    % Índice de códigos
	
	% Inicia la numeración habitual.
	\mainmatter
	
	\setstretch{1.5} % 1.5 línea de interlineado
	
	%%%%
	% CONTENIDO. CAPÍTULOS DEL TRABAJO - Añade o elimina según tus necesidades
	%%%%
	%%%%%%%%%%%%%%%%%%%%%%%%%%%%%%%%%%%%%%%%%%%%%%%%%%%%%%%%%%%%%%%%%%%%%%%%
% Plantilla TFG/TFM
% Universidad de A Coruña. Facultad de Informática
% Realizado por: Welton Vieira dos Santos
% Modificado: Welton Vieira dos Santos
% Contacto: welton.dossantos@udc.es
%%%%%%%%%%%%%%%%%%%%%%%%%%%%%%%%%%%%%%%%%%%%%%%%%%%%%%%%%%%%%%%%%%%%%%%%


\chapter{Desarrollo Artístico}
\section{Antecedentes}
En el año 2500 la Tierra era un yermo vacío y la humanidad la abandonó, 
mandando robots autoconscientes steampunk a explorar el espacio, ahora vivían colonizando el espacio. Figura \ref{fig:TierraDevastada}.

\begin{figure}[H]
	\centering
	\includegraphics[scale=0.80]{imagenes/terraDevastada.png}
	\caption{\label{fig:TierraDevastada}Tierra devastada}
\end{figure}

\section{Ambientación}
La historia se transcurre en una enorme y antigua nave en la que suceden cosas extrañas, de la que nuestro protagonista tendrá que huir.
ISS Manticore (Figura \ref{fig:ISSManticore}) es la nave misteriosa donde nuestro protagonista tenía que entregar un paquete.  

\begin{figure}[H]
	\centering
	\includegraphics[scale=0.80]{imagenes/2.png}	
	\caption{\label{fig:ISSManticore}ISS Manticore}
\end{figure}

\section{Historia}
Jackie (Figura \ref{fig:Jackie})es un cartero espacial que tiene que entregar un paquete en una misteriosa nave, pero cuando llega no hay nadie, y acaba encerrado en la misteriosa nave de la que tendrá que escapar. Cuando nuestro protagonista abre el paquete, verá que es un arma de fuego, la tendrá que usar para abrirse paso a través de la nave contra los misteriosos moradores que se vaya encontrando en ella.

\section{Personajes}
\begin{itemize}
	\item \textbf{Jackie:} Nuestro protagonista, no tiene muchas luces, pero sabe sobrevivir
	\begin{figure}[H]
		\centering
		\includegraphics[scale=0.95]{imagenes/personaje_1.png}	
		\caption{\label{fig:Jackie} Aspecto de Jackie}
	\end{figure}
	\item \textbf{Enemigos:} En esta enorme nave habitan unos seres que al parecer no les gusta mucho la compañía de otros, por lo que te atacaran si te cruzas en su camino.
	\item \textbf{Octopus:} Son unos seres extraños fruto de un experimento de crear inteligencia artificial utilizando cerebros biológicos. 
	\begin{figure}[H]
		\centering
		\includegraphics[scale=0.95]{imagenes/octopus.png}	
		\caption{\label{fig:Octopus}Aspecto de un Octopus}
	\end{figure}
	\item \textbf{Tortuga:} Alguien se olvidó de alimentar a las mascotas de la nave, pero por alguna razón se siguen moviendo,y parecen hambrientas. 
	\begin{figure}[H]
		\centering
		\includegraphics[scale=0.60]{imagenes/tortuga.png}	
		\caption{\label{fig:Tortuga}Aspecto de una Tortuga}
	\end{figure}
	\newpage
	\item \textbf{Spiderbots:} Alguien que se aburría en esta nave se le ocurrió la brillante idea de ponerle patas a las torretas, y parece que muy bien no salió. 
	\begin{figure}[H]
		\centering
		\includegraphics[scale=0.6]{imagenes/spiderbot.png}	
		\caption{\label{fig:Spiderbots}Aspecto de un Spiderbot}
	\end{figure}
\end{itemize}

\section{Jugabilidad}
Side scroller en el que completas niveles estilo metroidvania, en el que el 
personaje pelea:  
\begin{itemize}
	\item Utilizando un arma de fuego (escopeta - Figura \ref{fig:escopeta}),
	 en el que avanzamos de forma similar a juegos como “My Friend Pedro”
	   avanzando por diferentes niveles acabando con los enemigos(Figura 1.6),
	    con una estética de enemigos y de ambientación ligeramente steampunk.
\end{itemize}
\begin{figure}[H]
	\centering
	\includegraphics[scale=0.80]{imagenes/escopeta.png}	
	\caption{\label{fig:escopeta}Ejemplo de la escopeta}
\end{figure}

\begin{figure}[H]
	\centering
	\includegraphics[scale=0.45]{imagenes/5.png}	
	\caption{\label{fig:ambiente}Ejemplo de un posible escenario y la dinámica de disparos en tempo real}
\end{figure}

\section{Items para el personaje}
\begin{itemize}
	\item \textbf{Vidas:}.Para recuperar vida y ayudar al jugador e incentivarlo a explorar el escenario.(Figura \ref{fig:Ejemplo Vida})
	\begin{figure}[H]
		\centering
    	\includegraphics[scale=0.95]{imagenes/vida.png}	
		\caption{\label{fig:Ejemplo Vida}Imagen de item para recuperar a vida}
	\end{figure}
	\item \textbf{Wrench:} Items coleccionables necesarios para completar
			el segundo nivel en el que tienes que recoger piezas para reparar el motor de la nave y así también hacer al jugador explorar el escenario (Figura \ref{fig:Wrench}).
	\begin{figure}[H]
		\centering
		\includegraphics[scale=0.55]{imagenes/wrench.png}	
		\caption{\label{fig:Wrench}Imagen de wrench representada por una llave inglesa girando}
	\end{figure}
\end{itemize}

\section{Lugares}
\subsection{ISS Manticore}
La nave en la que se desarrolla el juego, está llena de misterios y seres extraños poco amistosos
tiene diferentes zonas y salas dentro,
en este juego se incluyen 3 salas, que corresponderían a 3 niveles.

\subsection{Sala de calderas de la nave}
El primer lugar al que llega el protagonista, en esta zona de la nave se encuentran todo el sistema del motor de la nave.

\subsection{Bodega de carga}
Es la bodega de carga de la nave, donde está el almacen y tiene un montón de extraños trastos viejos de la nave y muchas herramientas que serán de utilidad al protagonista.

\subsection{Puente de mando}
La última sala que el protagonista deberá atravesar, y así conseguir al fin huir de la nave.

\section{Guion}
\subsection{Fase 1 - Sala de calderas}
La escena empieza con el cartero intentando entregar ese supuesto paquete a la nave y al llegar cerca de la nave el personaje \textbf{Jackie} se da cuenta que en esa nave no hay nadie que pueda responder y el toma la decisión de entrar en la nave y el mismo se queda atrapado. 

Para salir de esa nave el tiene que buscar otra salida. Jackie decide abrir el paquete y se dentro del mismo se encuentra un arma y decide seguir hacia delante para intentar localizar otra salida.

\subsection{Fase 2 - Almacen}
Esa escena empezará con unos de los enemigos dañando un motor de la sala de calderas lo que obriga a nuestro personaje (Jackie) a intentar localizar piezas para reparar ese motor dañado avanzando a este siguien escenario que será el almacen de la bodega de carga.
Nuestro personaje encuentra herramientas y piezas al 
recorrer el camino que pueden ser utilizadas para reparar el motor.
Además de ir derrotando los enemigos que van apareciendo a medida que Jackie avanza.
Para terminar la fase, Jackie tiene que haber recogido todas las piezas que se encuentra en esa fase para reparar el motor.

\subsection{Fase 3 - Casco superior de nave}

Esa fase se empieza una batalla final con los últimos enemigos que protejen la mesa de mando donde nuestro personaje tiene que enchufar una tarjeta, que contiene un chip que permite Jackie efetuar un pedido de auxilio. Al momento que Jackie enchufa la tarjeta, la nave sufre un apagon debido daños causados anteriormente por los enemigos.

\section{Videojuego en 3D}

\subsection{Fases}
\subsubsection{Fase 1 - Motores exteriores de la nave}
Nuestro protagonista, tiene que salir a arreglar el motor de la nave para recuperar la electricidad y
 que se envíe la llamada de ayuda, así que tendrá que salir fuera de la nave (por lo que no hay gravedad) ,
  y con su pistola gancho tendrá que ir avanzando por la turbinas girando y demás elementos flotando
   hasta llegar al motor a arreglar 

\subsubsection{Fase 2 - Zonas de motores}
El protagonista se tiene que defender de enemigos mientras repara el motor.

\subsubsection{Fase 3 - Espacio exterior}
Una nave para salvarle aparece, nuestro protagonista tendrá que ir escalando entre basura espacial y cajas
(de la sala de carga) flotando en el espacio entre las dos naves para poder huir y alcanzar a la otra nave.
(las torretas de la nave de la que huye le pueden ir disparando mientras para que no huya).
	
	%%%%%%%%%%%%%%%%%%%%%%%%%%%%%%%%%%%%%%%%%%%%%%%%%%%%%%%%%%%%%%%%%%%%%%%%
% Plantilla TFG/TFM
% Universidad de A Coruña. Facultad de Informática
% Realizado por: Welton Vieira dos Santos
% Modificado: Welton Vieira dos Santos
% Contacto: welton.dossantos@udc.es
%%%%%%%%%%%%%%%%%%%%%%%%%%%%%%%%%%%%%%%%%%%%%%%%%%%%%%%%%%%%%%%%%%%%%%%%


\chapter{Desarrollo técnico}
\section{Videojuego en 2D}
\subsection{Descripción}ISS Manticore es un scroll lateral con ligeros tintes de metroidvania\footnote{Metroidvania: es un subgénero de juego basado en un concepto de plataformas no lineal con un mundo conectado que fomenta que el jugador lo explore} en el diseño de escenarios al tener algunos incentivos para desviarse del camino principal, pero dividido en niveles donde los jugadores comienzan cada fase en el extremo izquierdo de un escenario lineal.

Su objetivo es alcanzar la salida de la nave de la cual se ha quedado atrapado en el momento que ha intentado entregar un paquete en la misma.

El protagonista va avanzando y derrotando los enemigos que contiene en cada fase.
\subsection{Personajes}
Jackie, el protagonista del juego tendrá que completar distintas fases, en las que tendrá que superar distintos enemigos y acometer distintas tareas para alcanzar su objetivo, para ello el jugador tendrá que mover al protagonista empleando las flechas del teclado para avanzar, esquivar a los enemigos y recoger los coleccionables necesarios y disparando con un click de ratón a los distintos personajes que se interpongan en su camino.

\subsection{Enemigos}
Los enemigos se interponen en el camino del protagonista habitando los diferentes escenarios por los que avanza dañando al protagonista si este entra en contacto con los enemigos. 
\begin{itemize}
    \item \textbf{Octopus:} Estos pequeños seres extraños se encuentran levitando en una misma posición con el objetivo de interrumpir al protagonista en su camino para completar su cometido. Ejemplo en la Figura \ref{fig:Octopus}.
    \item \textbf{Tortugas:} Están en constante movimiento de un lado a otro y dificultan el transcurso del personaje, que debe abatirlos o esquivarlos, para no salir dañado y conseguir completar su misión. Ejemplo en la Figura \ref{fig:Tortuga}.
    \item \textbf{Spiderbots:} Son el enemigo más difícil de derrotar para el protagonista debido a que dispara proyectiles cada 10 segundos que pueden dañarlo si no es capaz de esquivarlos. Ejemplo en la Figura \ref{fig:Octopus}. 
\end{itemize}

\subsection{Diseño}
\subsubsection{Patrón estrategia}
Este patrón ha sido utilizado en la gran mayoría del código, con la idea de abstraer en clases una serie de estados y comportamientos y, a medida que se se van extendiendo otras clases de una forma bastante jerárquica, haciendo que el comportamiento sea cada vez más específico, por ejemplo el Menu principal del juego que está compuesto por varios componentes como imágenes, botones (estilo rolover) para seleccionar las opciones pertinentes.

\subsubsection{Patrón Singleton}
El patrón singleton está presente en la clase Director, GestorRecursos y tambien en las factorías de los personajes y de las fases del juego. Además de solamente permitir que se instancia solamente un objecto con el uso de clases internas en el caso del director, que solamente tiene uno en todo el proceso.

\subsubsection{Personajes}
La implementación de los personajes hereda de la clase MiSprite y va jerarquizando como se muestra en la Figura \ref{fig:Personaje}. Esa clase incorpora los elementos necesarios para almacenar las posiciones, sprites, velocidades y todos los elementos comunes como el scroll.

\begin{figure}[H]
	\centering
	\includegraphics[scale=0.30]{imagenes/Personaje.png}
	\caption{\label{fig:Personaje}Ejemplo de la jerarquía de la clase Personaje}
\end{figure}

Cada objeto del tipo personaje es instanciado por su factoría correspondiente con la intención de permitir un mejor mantenimiento y expansión del código futuramente.

Esos personajes están diferenciados por dos tipos básicos, uno jugable y otro los enemigos.

\subsubsection{Fases}
La implementación de la dinámica de fases es muy similar a la dinámica de los personajes como se puede apreciar en la Figura \ref{fig:Fases} concervando los métodos ``update'', ``eventos'' y ``dibujar'' de la clase Escena.

\begin{figure}[H]
	\centering
	\includegraphics[scale=0.30]{imagenes/Fases.png}
	\caption{\label{fig:Fases}Ejemplo de la jerarquía de la clase Fase}
\end{figure}

\subsubsection{Escena}
La clase Escena define toda las estructura visual del juego, donde desde ahí se puede controlar los menus, personajes, transiciones y etc.

La clase Escena posee tres métodos principales:
\begin{itemize}
	\item eventos: Encargado de leer los eventos producidos por la interacción del usuario con el sistema.
	\item update: Encargado de actualizar el modelo de escena en cuestión basado en los eventos producidos durante la interacción del usuario.
	\item dibujar: Encargado de dibujar los elementos de la parte visual del juego.
\end{itemize}

\subsubsection{Director}
El director es el encargado de ejecutar la escena pertinente, que puede ser un menú o las fases correspondientes de las determinadas etapas del juego. La Figura \ref{fig:Director} muestra su estructura.

\begin{figure}[H]
	\centering
	\includegraphics[scale=0.30]{imagenes/Director.png}
	\caption{\label{fig:Director}Ejemplo de la jerarquía de la clase Director}
\end{figure}

Como se puede observar, director possee una clase interna.

\subsubsection{Gestor de recursos}
Como su propio nombre dice, es el encargado de suministrar los recursos necesarios para la una buena interacción con el juego. La Figura \ref{fig:GestorRecursos}

\begin{figure}[H]
	\centering
	\includegraphics[scale=0.30]{imagenes/GestorRecursos.png}
	\caption{\label{fig:GestorRecursos}Ejemplo uml del gerenciador de recursos}
\end{figure}

\subsection{Escenas}

La transición entre escenas se muestra en el diagrama de la Figura \ref{fig:TrasicionEscenas}. 

\begin{figure}[H]
	\centering
	\includegraphics[scale=0.35]{imagenes/transicionEscenas.jpeg}
	\caption{\label{fig:TrasicionEscenas}Diagrama de transición de las escenas}
\end{figure}

\subsubsection{Menú principal}
En esta primera pantalla (Figura \ref{fig:EjemploMenuPrincipal}), que se muestra al jugador nada más ejecutar el juego, se le permite elegir entre comenzar con la historia, visualizar los créditos, lo que enviaría al usuario a la pantalla de Créditos, o la opción de salir, que cerraría el juego. 

\begin{figure}[H]
	\centering
	\includegraphics[scale=0.50]{imagenes/EjemploMenuPrincipal.png}
	\caption{\label{fig:EjemploMenuPrincipal}Ejemplo del menu principal}
\end{figure}

\subsubsection{Historia}
Esta pantalla (Figura \ref{fig:EjemploMenuHistoria})muestra un breve resumen de la historia y la leyenda del juego, en la que se puede ver la configuración inicial que contiene la imagen del personaje, el número de vidas inicial, el número de proyectiles con el que cuenta antes de comenzar la aventura (con miras a añadir más adelante objetos de munición y diferentes armas o proyectiles). Desde esta pantalla se puede volver al Menú principal pulsando la tecla Esc o pasar a la Escena 1 pulsando la tecla Enter. 

\begin{figure}[H]
	\centering
	\includegraphics[scale=0.50]{imagenes/EjemploMenuHistoria.png}
	\caption{\label{fig:EjemploMenuHistoria}Ejemplo del apartado historia del menu principal}
\end{figure}

\subsubsection{Fase 1 - Sala de calderas}
La escena comienza con el protagonista en la sala de calderas, este emprende su aventura con tres vidas, que podrá aumentar o disminuir segundo se desenvuelva, nada más arrancar aparece el primer enemigo, un spiderbot al que tendremos que derrotar, si seguimos avanzando nos encontramos con el segundo enemigo, en este caso un octopus, y también con el primer botiquín, que nos permitirá ganar una vida, continuando con la aventura nos encontraremos con más enemigos a los que tendremos que derrotar, para llegar al final de esta escena. 

Para pasar a la siguiente escena deberemos continuar andando al llegar al final del escenario.  

En ese nivel nuestro personaje encontrará con distintos enemigos que una vez superados se terminará esa fase del del juego. La Figura \ref{fig:EjemploEscena_1} muestra un ejemplo de la escena 1. 

\begin{figure}[H]
	\centering
	\includegraphics[scale=0.50]{imagenes/EjemploEscena_1.png}
	\caption{\label{fig:EjemploEscena_1}Ejemplo de la fase 1}
\end{figure}

\subsubsection{Fase 2 - Almacen}
En esta segunda escena el protagonista se encuentra con 3 vidas en el almacén, donde tendrá que recoger todas las herramientas (un total de 11) que se va a ir encontrando para completar con éxito esta etapa, también se encontrará con botiquines que le permitirán recuperar vidas, las cuales le podrán ser arrebatadas por los enemigos que entorpecerán su camino, tendrá que hacer frente a 7 tortugas, 6 spiderbots y 7 octopus. Si continúa andando al llegar al final del escenario, pasará a la escena 3. En la Figura \ref{fig:EjemploEscena_2} presenta un ejemplo de la misma. 

\begin{figure}[H]
    \centering
    \includegraphics[scale=0.50]{imagenes/EjemploEscena_2.png}
    \caption{\label{fig:EjemploEscena_2}Ejemplo de la fase 2}
\end{figure}
\subsubsection{Fase 3 - Puente de mando de la nave}

En la tercera y última escena el jugador aparece en el Puente de mando con 3 vidas, al igual que en las escenas anteriores, debe llegar al final del escenario con vida intentando destruir a los enemigos que se encuentre a su paso, un total de 20: 7 tortugas, 7 octopus y 6 spiderbots. Si consigue llegar con vida al final de esta escena habrá completado la aventura y llegará a la pantalla de Victoria. En la Figura \ref{fig:EjemploEscena_3} presenta un ejemplo de la misma.

\begin{figure}[H]
    \centering
    \includegraphics[scale=0.50]{imagenes/EjemploEscena_3.png}
    \caption{\label{fig:EjemploEscena_3}Ejemplo de la fase 3}
\end{figure}

\subsubsection{Créditos}
Pantalla en la que se visualiza la lista de nombres de los desarrolladores y del product owner. Desde esta pantalla se permite al jugador volver al menú principal pulsando la tecla Esc. Ejemplo en la Figura \ref{fig:EjemploCreditos}.

\begin{figure}[H]
	\centering
	\includegraphics[scale=0.50]{imagenes/ApartadoCreditos.png}
	\caption{\label{fig:EjemploCreditos}Ejemplo del apartado créditos del menú principal}
\end{figure}

\subsubsection{Victoria}
En esta pantalla se muestra un aviso de que el jugador ha conseguido ganar y terminar el juego. Para volver al menú principal se puede pulsar la tecla Esc. Ejemplo en la Figura \ref{fig:EjemploVictoria}.

\begin{figure}[H]
	\centering
	\includegraphics[scale=0.50]{imagenes/victoria.png}
	\caption{\label{fig:EjemploVictoria}Ejemplo de la escena de victoria}
\end{figure}

\subsubsection{Derrota}
Aquí se informa al jugador de que ha sido derrotado, tras haberse quedado sin vidas. Desde esta pantalla se puede volver al menú principal pulsando a tecla Esc. Ejemplo en la Figura \ref{fig:EjemploDerrota}.

\begin{figure}[H]
	\centering
	\includegraphics[scale=0.60]{imagenes/derrota.png}
	\caption{\label{fig:EjemploDerrota}Ejemplo de la escena de derrota}
\end{figure}

\subsubsection{Pausa}
Aquí se informa al jugador de que el juego sido pausado despues de precionar la tecla de escape (esc). Desde esta pantalla se puede volver al menú principal pulsando a tecla Esc o seleccionar la opción \textbf{Menu Principal}. Ejemplo en la Figura \ref{fig:EjemploPausa}.

\begin{figure}[H]
	\centering
	\includegraphics[scale=0.50]{imagenes/Pausa.png}
	\caption{\label{fig:EjemploPausa}Ejemplo de pausar el juego}
\end{figure}

\subsection{Aspectos destacables y detalles de su implementación}

\subsubsection{Creacción de las fases}
Un aspecto del proyecto que cabe destacar es la forma en que se crean los escenarios y se posicionan los enemigos e items, esta parte se realiza a través de una arquitectura dirigida por datos, donde cada escenario está representado por una matriz de números en un fichero .txt, donde cada número representa un tipo de bloque diferente y los ceros que no hay ningún bloque. 

Los tipos de bloque (“tiles”) están representados en ficheros json, con un array de elementos, cada uno con un identificador que corresponde al valor numérico en la matriz. El nombre de la hoja de sprites que contiene el sprite del bloque y la posición en la hoja de sprites. 

La colocación de los enemigos y los items es parecida:  

Para cada enemigo y para cada item tenemos un fichero json formado por un array de elementos. Estos elementos contienen un nº identificador, la posición en el eje x y la posición en el eje y. 

Después estos datos son leídos por las funciones: 

“readMatrix” : Para leer a matriz que representa o escenario 

“readPosition”: Para leer de las posiciones de los items y los enemigos 

“ReadTileReferences”: para leer o json que define los bloques (“tiles”) 

\subsubsection{Dinámicas de ataques}

Tanto para los disparos del personaje, como para los enemigos y los items, para que no se crearan y destruyeran objetos de forma que acabara pasando el recolector de basura provocando que se atasque el juego, todos estos elementos son creados al iniciarse el nivel, y al tener que ser destruidos son cambiados del grupo al que corresponden a un grupo que no se actualiza, por lo que se mantienen sus referencias de modo que el juego no se atasca borrandolos, pero tampoco se pierde tiempo de ejecución actualizándolos. 

Para los disparos, creamos un array de un nº máximo de balas suficientes para que se puedan reutilizar sin que el jugador se entere, este array pertenece al personaje que las dispara y sus balas al inicializarse pertenecerán al grupo de balas que no se actualizan, así, cada vez que dispare llamará a una función que le reinicia todos los atributos a la bala y esta se añade al grupo de balas que sí se actualizan. 

En el protagonista al disparar se calculará el ángulo de disparo del ratón respecto al protagonista. 

\section{Videojuego en 3D}

\subsection{Descripción}
ISS Manticore 3D es un First Person Shooter(FPS), donde el protagonista debe superar diferentes desafios hasta alcanzar su objetivo. Para ello dispondrá de armamento del que se podrá ayudar para derrotar a los distintos enemigos que tratarán de entorpecer su paso. Al comienzo de la aventura tendrá que desplazarse hasta encontrar las dos palancas que le permitiran pasar a la siguiente fase, durante este trayecto irá eliminando a los enemigos que se crucen en su camino. Una vez completado este desafío, el personaje tendrá que arreglar un motor, para ello ha de permanecer en la zona en que se encuentra el motor durante un tiempo suficiente, esto no va a resultarle tarea fácil puesto que habrá multiples enemigos tratando de dañarlo. Finalmente tendrá que conseguir llegar a una pequeña nave de emergencia, ayudandose de las plataformas que se encontrará por el espacio y le permitirán alcanzar la nave. 

\subsection{Personajes}

En esta entrega 3D, solamente tendrá un personaje principal, es decir, el mismo aventurero y cariñoso Jackie.

Como en la entrega 2D, el protagonista tendrá que completar distintas fases, en las que tendrá que superar distintos enemigos y acometer distintas tareas para alcanzar su objetivo, para ello el jugador tendrá que mover al protagonista empleando las flechas del teclado para avanzar, esquivar a los enemigos y recoger los coleccionables necesarios y disparando con un click de ratón a los distintos personajes que se interpongan en su camino. La dinámica es la misma a excepción de que ahora se pondrá acompañar el personaje en primera persona y que el manejo de la visión de Jackie se efectuará desde el ratón.

\subsection{Enemigos}

\begin{itemize}
	\item \textbf{Bot Enemy:} Enemigo a distancia que dispara y causa daño
	\begin{figure}[H]
		\centering
		\includegraphics[scale=0.60]{imagenes/BotEnemy.png}
		\caption{\label{fig:BotEnemy}Ejemplo del Bot Enemy}
	\end{figure}
	\item \textbf{Spider Bot Enemy:} Enemigo a melee \footnote{Melee: expresión para indicar que hay una lucha cuerpo a cuerpo}.

	\begin{figure}[H]
		\centering
		\includegraphics[scale=0.60]{imagenes/SpiderBotEnemy.png}
		\caption{\label{fig:SpiderBotEnemy}Ejemplo del Spider Bot Enemy}
	\end{figure}
\end{itemize}



\subsection{Diseño}

\subsubsection{Diseño del HUD - Patrón Observador}
El HUD consta de varios elementos:
\begin{itemize}
	\item \textbf{Tiempo:} Indica cuanto tiempo se lleva en el nivel.
	\item \textbf{Puntos:} Indica la puntuación del jugador matando enemigos.
	\item \textbf{Vidas:} Indica cuanta vida le queda al jugador.
	\item \textbf{Objetivo:} Explica al jugador cual es su objetivo actual, diciéndole a donde tiene que ir o qué tiene que hacer.
	\item \textbf{Munición:} Indica cuanta munición le queda al arma del jugador.
	\item \textbf{\% Reparación:} Exclusivo del 2º nivel, indica el porcentaje de la reparación del motor llevada a cabo.
\end{itemize}

En la Figura \ref{fig:PantallaHUD3D} muestra un ejemplo de visualización del HUD en pantalla.

\begin{figure}[H]
	\centering
	\includegraphics[scale=0.45]{imagenes/PantallaHUD3D.png}
	\caption{\label{fig:PantallaHUD3D}Ejemplo de la visualización del HUD en pantalla}
\end{figure}

El menú de pausa consta de estos elementos:
\begin{itemize}
	\item \textbf{Continuar:} Para quitar el menú de pausa y seguir con el juego
	\item \textbf{Menu principal:} Para volver al menú principal.
	\item \textbf{Salir:} Para salir del juego.
\end{itemize}
El menú de derrota, que aparece cuando el jugador pierde toda su vida consta de estos elementos:
\begin{itemize}
	\item \textbf{Reiniciar nivel:} Para reiniciar el nivel actual.
	\item \textbf{Menu principal:} Para volver al menú principal.
	\item \textbf{Salir:} Para salir del juego.
\end{itemize}

\subsubsection{Diseño de las escenas}
La 1º escena utiliza como base de la escena el prefab de una de las naves disponibles en el paquete "Federation Corvette F3", donde el jugador se moverá por su superficie, aparte se han añadido varios obstáculos y paredes en el escenario. Se han repartido enemigos por el escenario para dificultar el avance del jugador, también se han añadido dos palancas, una en el extremo de la nave, con una luz potente para que se fácil saber hacia donde dirigirse, y la 2º en una plataforma en el medio del escenario junto a la puerta que abren estas dos en conjunto.

En la 2º escena, que se desarrolla en el interior de la nave, el escenario está compuesto por unos pasillos en los que se encuentra el motor de la nave en el centro. En uno de los pasillos con los que intersecciona hay una puerta medio subida por donde acceden enemigos para perseguir al jugador, pero este no puede pasar por ese hueco. Al otro lado de la misma se spawnean los enemigos sin que el jugador lo vea. En uno de los extremos del pasillo, para fomentar la exploración, que el jugador se mueva por el escenario y facilitar el combate, se han colocado un spawner de un ítem de vida y un spawner de un ítem de munición.

La 3º escena, se desarrolla en el mismo lugar que el 1º nivel, pero esta vez se comienza donde el jugador entraba al 2º nivel desde el 1º, en el centro del mapa. Se han colocado plataformas estáticas y móviles que el jugador tendrá que utilizar para llegar a la pequeña nave que se encuentra posada sobre uno de las torres de la nave grande que hace de escenario. Para este nivel el script de movimiento de las plataformas que se creó también es capaz de rotarlas, pero utilizar esto en plataformas que empleará el jugador se acabó descartando porque dificultaba mucho al jugador saltar correctamente de una plataforma a otra.

\subsubsection{Zona de reparación del motor}
Para la reparación del motor, se ha creado un objeto que es un cubo invisible que se puede atravesar, el área de reparación, con un script que sigue el patrón observer, comprobando si el jugador se encuentra dentro de él o no. 

Para mostrar en la HUD el porcentaje de la reparación se han creado en el script \textit{Gamesystem.cs} las funciones ``MostrarReparacion'' (que sirve para indicar que se active el objeto del HUD que muestra el porcenaje de reparación) y ``UpdateReparacion'' (que actualiza el número que indica el porcentaje), que a su vez llaman a sus homónimos en la clase GameSystemInfo que se encuentran en el objeto GameUI, inicializado directamente por el GameSystem. 

\subsubsection{Mostrar el objetivo actual}
Para mostrar los objetivos el funcionamiento, con las funciones ``MostrarObjetivo'' y ``UpdateObjetivo'', estas funciones son llamadas al GameSystem por: 

En el 1º nivel por ``LeverSystem'', el sistema encargado del funcionamiento de las palancas, al iniciarse indicará que el objetivo es activar las palancas del escenario y dónde están localizadas. Y mediante el patrón observador, comprobará si el estado de todas las palancas es que estén activadas, entonces al haber activado todas las palancas se volverá a llamar a ``UpdateObjetivo'' para que actualice el objetivo diciéndole al jugador que tiene que atravesar el pasillo que ha abierto. 

En el 2º nivel, ``RepairZone'', el sistema encargado del funcionamiento del porcenataje de reparación, al iniciarse indicará que el objetivo es reparar el motor. Y mediante el patrón observador, comprobará si el estado del motor es que esté reparado, y entonces volverá a llamar a ``UpdateObjetivo'' para que actualice el objetivo diciendole al jugador que tiene que volver por donde vino.

En el 3º nivel, simplemente un script al iniciarse indicará que el objetivo es subir por las plataformas para llegar a una nave en la que huir, solo hay un objetivo en este nivel, por lo que no necesitará actualizarse más.

\subsubsection{Enemigos}
Respecto a los enemigos en común, comentar que utilizan el patrón plantilla siendo prefabs de Unity con varias opciones configurables. A continuación, se pueden ver los aspectos de cada uno:
\begin{itemize}
	\item \textbf{BotEnemy:} Este prefab de enemigo, parte del modelo del paquete Enemy Robots. Para este se ha creado un Script con nombre RobotAI, el cual permitirá al enemigo realizar diversas acciones. Por defecto, se moverá a través de los puntos marcados como pivots circularmente patrullando, esto permite al enemigo tener interacción propia con el sistema en el tiempo que no detecta al enemigo. Este enemigo detectará al jugador una vez este entre dentro de un rango de distancia, configurable a través del prefab, y lance un LineCast para detectar si hay algún otro objeto entre medias, así se evita que el jugador piende que el enemigo tiene una ventaja por el simple hecho de detectarlo sin llegar a verlo. Una vez detectado, el enemigo empezará a perseguir al jugador, manteniendo un mínimo de distancia con él. Consta, así mismo, de un contador entre ataques para disparar cada el tiempo indicado. Finalmente, en los ataques instancia un objeto EnemyBullet, al cual asigna un movimiento, este al colisionar con el jugador restará un punto de vida, comentar también que se ha tomado la decisión de que los disparos del enemigo sean un proyectil, y no hitscan (disparo inmediato sin tiempo de viaje), con un movimiento más lento y retardo a la hora de apuntar para darle ventaja al jugador y que no sienta que los enemigos están haciendo trampa al calcular donde atacar con precisión e inmediatez.
	\item \textbf{PA\_Warrior Variant:} Este prefab de enemigo, parte del prefab y animaciones PA\_Warrior del paquete SciFi Enemies and Vehicles. Nuevamente se ha creado un Script para dichos enemigos, estes incluyen el comportamiento inteligente dado por Unity para navegar hasta la posición del jugador. Una vez más tiene un contador para medir el tiempo entre ataques. Una vez se complete el tiempo de ese contador, si se encuentra en un rango suficientemente cercano del jugador iniciarán la animación de ataque y provocarán daño en el jugador. Se ha tomado la decisión de que los enemigos provoquen daño reducido al jugador, pero como persiguen y atacan muchos juntos son un tipo de enemigo más diverso. Mencionar, como patrón utilizado, la máquina de estados proporcionada por Unity para la animación. Vease Figura \ref{fig:MaquinaEstados}.
\end{itemize}

\begin{figure}[H]
	\centering
	\includegraphics[scale=0.65]{imagenes/MaquinaEstados.png}
	\caption{\label{fig:MaquinaEstados}Ejemplo de la visualización del HUD en pantalla}
\end{figure}

\subsubsection{Puntuación}
Cada vez que se elimina a un enemigo se suma un punto a la puntuación del jugador, que puede observar en la parte superior izquierda del HUD. Este sistema sigue un patron comando, pues cada vez que se ejecuta este ordena al HUD que actualice la puntuación.

\subsubsection{Sistema de palancas}
En cuanto al sistema de palancas, se utiliza el prefab de la palanca y su correspondiente animación, visible cuando el personaje la acciona. Se ha modificado el script propio de la palanca para obtener Lever con el comportamiento deseado. Además, se ha creado otro script nuevo para el sistema, denominado LeverSystem, que permite que, al añadir a la escena tantas palancas como se deseen, solo se abra la puerta cuando todas y cada una de ellas estén accionadas.

\subsubsection{Sistema de vidas}
El sistema de vidas gestiona la salud del personaje, cuando una bala de un bot enemy golpea al personaje, este pierde una vida. En cambio, cuando es un spiderbot el que alcanza al personaje, este se queda sin un cuarto de vida. También permite al personaje recuperar vida cuando este recoge un ítem de vida. Cuando el personaje se queda sin vidas el sistema muestra el menú de Game Over donde el jugador puede escoger entre reiniciar el nivel en el que se encuentra, volver al menú principal o abandonar el juego.

\subsubsection{Items de recuperación de munición y vida}
Para estos items se ha utilizado utilizado unos modelos con forma de maletines, para la munición se utiliza el script ``Ammo Box'' del ``Creator Kit: FPS'', y recupera 10 unidades de munición.

En el caso del item de vida, se creó un script con un funcionamiento similar al de munición, al colisionar el jugador con el item este llama a la función de recuperar vida y se destruye.

Estos dos scripts siguen un patrón comando pues ordenan actualizar los datos de munición (Figura \ref{fig:EjemploMunicion3D}) y de vida (Figura \ref{fig:EjemploVida3D}).

\begin{figure}[H]
	\centering
	\includegraphics[scale=0.65]{imagenes/EjemploMunicion3D.png}
	\caption{\label{fig:EjemploMunicion3D}Ejemplo de un recolectable de munición}
\end{figure}

\begin{figure}[H]
	\centering
	\includegraphics[scale=0.65]{imagenes/EjemploVida3D.png}
	\caption{\label{fig:EjemploVida3D}Ejemplo de un recolectable de vida}
\end{figure}

\subsubsection{Objeto de cambio de escena}
Para el cambio de escena de cada nivel se ha utilizado un prefab (por lo que sigue el patrón plantilla) que es un cubo invisible con el script ``ChangeScene'', este se activa cuando el jugador entra en contacto con el objeto cargando el escenario indicado. Al ordenar cambiar de escenario siguen un patrón comando.

\subsubsection{Menu de pausa}
Se utilizó el menu de pausa proporcionado por ``Creator Kit: FPS'' pero cambiando el botón de seleccion de nivel por un botón para ir menú principal.

\subsubsection{Jugador, controlador del jugador y sistema de disparos}
Para estos 3 elementos se utilizaron los proporcionados por "Creator Kit: FPS" y no se realizó ningún cambio destacable en ellos.

\subsection{Escenas}

La transición entre escenas se muestra en el diagrama de la Figura \ref{fig:TrasicionEscenas3D}. 

\begin{figure}[H]
	\centering
	\includegraphics[scale=0.75]{imagenes/transicionEscenas3D.png}
	\caption{\label{fig:TrasicionEscenas3D}Diagrama de transición de las escenas}
\end{figure}

\subsubsection{Menú principal}
El menú principal consta de 3 botones como se muestra en la Figura \ref{fig:MenuPrincipal3D1}.
\begin{figure}[H]
	\centering
	\includegraphics[scale=0.35]{imagenes/MenuPrincipal3D1.png}
	\caption{\label{fig:MenuPrincipal3D1}Ejemplo del Menu Principal}
\end{figure}

Detalles de cada opción del menu presentado en la Figura \ref{fig:MenuPrincipal3D1}:
\begin{itemize}
	\item \textbf{Nueva Partida:} Activa una segunda ventana en el menú con una explicación de los controles del juego y un botón ``Continuar al juego'' que lleva al nivel 1 como se muestra en la Figura \ref{fig:MenuPrincipalNuevaPartida}.
	\begin{figure}[H]
		\centering
		\includegraphics[scale=0.40]{imagenes/MenuPrincipalNuevaPartida.png}
		\caption{\label{fig:MenuPrincipalNuevaPartida}Apartado Nueva partida del Menu Principal}
	\end{figure}
	\item \textbf{Créditos:} Activa una segunda ventana en el menú con los créditos del juego y un botón ``cerrar'' para cerrar esta ventana como se muestra en la Figura \ref{fig:MenuPrincipalCreditos}.
	\begin{figure}[H]
		\centering
		\includegraphics[scale=0.35]{imagenes/MenuPrincipalCreditos.png}
		\caption{\label{fig:MenuPrincipalCreditos}Apartado Crédito del Menu Principal}
	\end{figure}
	\item \textbf{Salir:} Cierra el juego.
\end{itemize}


\subsubsection{Menú de derrota}
El menú de derrota consta de un texto informativo para indicar que se ha terminado el juego con una frase ``Game Over'' y tres opciones al usuario (Figura \ref{fig:MenuDerrota}):
\begin{itemize}
	\item \textbf{Reiniciar Nivel:} Devuelve al jugador al inicio del nivel del cual ha sido derrotado.
	\item \textbf{Menu Principal:} Devuelve al jugador al Menu Principal del juego.
	\item \textbf{Salir}: Permite el jugador salir del juego.
\end{itemize}

\begin{figure}[H]
	\centering
	\includegraphics[scale=0.40]{imagenes/MenuDerrota.png}
	\caption{\label{fig:MenuDerrota}Ejemplo de la pantalla de derrota}
\end{figure}

\subsubsection{Menú de victoria}
El menú de victoria consta de un texto informativo para indicar que se ha completado el juego (Figura \ref{fig:MenuVictoria3D}), y consta de 1 botón:
\begin{itemize}
	\item \textbf{Reiniciar Juego:} Devuelve al jugador al Menú Principal (Figura \ref{fig:MenuPrincipal3D1}), donde podrá escoger que acción tomar.
\end{itemize}

\begin{figure}[H]
	\centering
	\includegraphics[scale=0.85]{imagenes/MenuVictoria3D.png}
	\caption{\label{fig:MenuVictoria3D}Ejemplo de la pantalla de victoria}
\end{figure}

\subsubsection{Menú de pausa}
El menú de pausa consta de tres botones para que el jugador pueda elegir una de las opciones (Figura \ref{fig:MenuPausa}):
\begin{itemize}
	\item \textbf{Continuar:} Permite que el jugador pueda continuar la partida.
	\item \textbf{Menu Principal:} Permite que el jugador pueda regresar al Menu Principal del juego.
	\item \textbf{Menu Salir:} Permite que el jugador pueda salir del juego.
\end{itemize}


\begin{figure}[H]
	\centering
	\includegraphics[scale=0.85]{imagenes/MenuPausa.png}
	\caption{\label{fig:MenuPausa}Ejemplo de la pantalla de pausa}
\end{figure}

\subsubsection{Escena 1}
La primera fase empieza en un extremo del escenario, el personaje debe recorrerlo hasta encontrar las dos palancas (Figura \ref{fig:Palanca}) que debe accionar para conseguir pasar al siguiente nivel, durante el trayecto por el escenario el personaje ha de eliminar a los enemigos que intentarán dificultar su devenir.

\begin{figure}[H]
	\centering
	\includegraphics[scale=0.75]{imagenes/Palanca.png}
	\caption{\label{fig:Palanca}Ejemplo de la palanca}
\end{figure}

La única forma de superar esa fase es accionando dos palancas, que se encuentran distribuidas por el escenario y permiten abrir la puerta que da acceso al siguiente nivel.

Para el diseño de este escenario se han utilizado los Prefabs que proporcionan los siguientes paquetes:
\begin{itemize}
	\item \textbf {Creator kit - FPS:} Se utiliza como base para el personaje,su controlador, el sistema de disparos, el menú de pausa y el HUD
	\item \textbf{SpaceSkies Free:} Se utiliza uno de sus assets como skybox
	\item \textbf{Federation Corvette F3:} Se utiliza como base del escenario
	\item \textbf{Interactive Physical Door Pack:} Se utiliza como base para crear las palancas
	\item \textbf{Simple Heart Health System:} Se utiliza como base para crear el sistema de vidas
	\item \textbf{Enemy Robots:} Se utilizan para los modelos de los enemigos
\end{itemize}


\subsubsection{Escena 2}
Para completar esta fase, será necesario arreglar un generador (Figura \ref{fig:Generador}) que se encuentra en el centro del escenario, en el cual aparecerán constantemente enemigos desde una zona no acesible que dificultarán al personaje arreglar el problema en el generador. Para realizar esta tarea se ha optado por la mecánica de capturar el objetivo lo cual sucede de forma pasiva mientras el jugador se mantenga en el área cercana. Para facilitar dicha tarea, en un pasillo externo el jugador podrá encontrar cajas de munición y vida para recuperarse y poder continuar luchando, dichos objetos apareceran cada 40 segundos. Una vez se complete, el personaje tendrá que volver por el mismo camino por el cual llegó para pasar a la última fase.

\begin{figure}[H]
	\centering
	\includegraphics[scale=0.75]{imagenes/Generador.png}
	\caption{\label{fig:Generador}Ejemplo del generador}
\end{figure}

Para el diseño de este escenario se han utilizado los Prefabs que proporcionan los siguientes paquetes:
\begin{itemize}
	\item \textbf{Creator kit - FPS:} Se utiliza como base para el personaje,su controlador, el sistema de disparos, el menú de pausa y el HUD
	\item \textbf{SpaceSkies Free:} Se utiliza uno de sus assets como skybox
	\item \textbf{Federation Corvette F3:} Se utiliza como base del escenario
	\item \textbf{Simple Heart Health System:} Se utiliza como base para crear el sistema de vidas
	\item \textbf{Sci-Fi Styled Modular Pack:} Se utiliza como base para la construcción del escenario encima de la nave, con el generador y los pasillos
	\item \textbf{SciFi Enemies and Vehicles:} Se utiliza para el modelo del enemigo con sus animaciones de ataque y persecución
	\item \textbf{Boxes pack:} Se utiliza para las cajas de munición y vida
\end{itemize}


\subsubsection{Escena 3}
En esta fase el objetivo es huir de esa nave para llegar a otra pequeña nave de emergencia que se encuentra localizada en una de las torres de ese escenario. El único inconveniente que el personaje va encontrar es que él mismo tiene que buscar una forma de acceder a dicha nave de emergencia, que está localizada en un plano distinto al del protagonista.

El personaje encontrará cajas flotantes que él mismo podrá utilizar saltando sobre ellas. Una vez llegado al objetivo, el personaje entrará en dicha nave de emergencia, esa fase se dará por concluida y se dará paso a la escena del menú de victoria.

Para el diseño de este escenario se han utilizado los Prefabs que proporcionan los siguientes paquetes:
\begin{itemize}
	\item \textbf{Creator kit - FPS:} Se utiliza como base para el personaje,su controlador, el sistema de disparos, el menú de pausa y el HUD
	\item \textbf{SpaceSkies Free:} Se utiliza uno de sus assets como skybox
	\item \textbf{Federation Corvette F3:} Se utiliza como base del escenario
	\item \textbf{Sci-Fi Styled Modular Pack:} Se utiliza para los modelos de las plataformas móviles
	\item \textbf{Simple Heart Health System:} Se utiliza como base para crear el sistema de vidas
	\item \textbf{Destructor Spaceship:} Se utiliza para el modelo de nave a la que tiene que llegar el personaje
\end{itemize}

\subsection{Aspectos destacables y detalles de su implementación}

\subsubsection{Plataformas móviles}
Para la creación de plataformas móviles se ha creado un prefab que permite a través del inspector controlar el movimiento y/o rotación de la plataforma a la que este asignada. Para ambas características se permite indicar las velocidades. En lo que respecta al movimiento este se basa en una serie de puntos a través de los cuales se irá moviendo y se pueden configurar a través del inspector. Implementan el patrón plantilla al ser prefab.

Además, cabe destacar que se hace uso de la herramienta ProBuilder de Unity para crear varios objetos de las escenas 1 y 3, empleando los prefabs disponibles, y también para crear un objeto con paredes invisibles que sirve para delimitar los escenarios.
	
	\chapter{Manual del juego en 2D}

\section{Estructura de las escenas}

La estructura del videojuego es la siguiente: 

\begin{figure}[H]
	\centering
	\includegraphics[scale=0.50]{imagenes/Estructura_escena.png}
	\caption{\label{fig:Estructura_escena}Estructura de las escenas del video juego}
\end{figure}

\section{Controles}

Las flechas del teclado permiten que el jugador se mueva por la pantalla en las direcciones de las flechas. Figura \ref{fig:flechasTeclado}.

\begin{figure}[H]
	\centering
	\includegraphics[scale=0.50]{imagenes/flechas_teclado.png}
	\caption{\label{fig:flechasTeclado}Teclas de movimiento del personaje}
\end{figure}

El ratón permite que el jugador apunte y dispare los proyectiles a sus enemigos con click izquierdo.

\begin{figure}[H]
	\centering
	\includegraphics[scale=0.50]{imagenes/raton.png}
	\caption{\label{fig:raton}Raton y click izquierdo}
\end{figure}

La tecla escape permite al jugador acceder al menú de pausa
\begin{figure}[H]
	\centering
	\includegraphics[scale=0.50]{imagenes/escape.png}
	\caption{\label{fig:escape}Tecla escape}
\end{figure}

\section{Ejecución del juego}
Para ejecutar el juego es preciso tener instaladas la librería pygame, tras eso, una vez posicionados dentro del directorio del proyecto, ejecutamos el juego con el comando: 

> python3 main.py 

\section{Navegación del Menu Principal}
Al iniciarse el juego aparece el menú de inicio con las siguientes opciones: 

\begin{figure}[H]
	\centering
	\includegraphics[scale=0.50]{imagenes/EjemploMenuPrincipal.png}
	\caption{\label{fig:EjemploMenuPrincipal}Ejemplo del menu principal}
\end{figure}

Para desplazarse entre las distintas opciones se utilizan las flechas del teclado y para seleccionarla se pulsa Enter.  

A continuación, se detallan cada una de ellas: 
\begin{itemize}
	\item \textbf{Historia:} muestra la leyenda del juego, con la configuración inicial en la que aparece la imagen del personaje, la cantidad de vida con la que comienza el jugador y la cantidad de munición (para como ya se mencionó implementar en el juego un sistema de munción y armas complejo en el futuro) 
	\item \textbf{Créditos:} muestra la lista de nombres de los desarrolladores y del product owner.  
	\item \textbf{Salir:} cierra la ventana del juego. 
\end{itemize}



Tanto desde la pantalla de la leyenda como desde los créditos se puede retroceder al menú pulsando la tecla Esc. 


Cuando da comienzo la historia van apareciendo los diferentes niveles que el jugador debe superar, si este llega al final aparecerá un indicativo demostrando que ha finalizado el juego. Si por el contrario el jugador pierde todas sus vidas por el camino aparecerá otro indicativo manifestando la derrota. Tras terminar el juego, tanto con éxito como si no, es necesario pulsar la tecla Esc para volver al menú principal.   	
	\chapter{Manual del juego en 3D}

\section{Estructura de las escenas}

La estructura del videojuego es la siguiente: 

\begin{figure}[H]
	\centering
	\includegraphics[scale=0.50]{imagenes/estructuraEscena3D.png}
	\caption{\label{fig:EstructuraEscena3D}Estructura de las escenas del video juego}
\end{figure}

\section{Controles}

Las teclas WASD y las flechas del teclado permiten que el jugador se mueva por el escenario en las direcciones de las flechas. Para manejar la cámara, es decir, lo que ve el personaje se utiliza el ratón para ese cometido. Figura \ref{fig:flechasTeclado}.

\begin{figure}[H]
	\centering
	\includegraphics[scale=0.40]{imagenes/teclasWASD.png}
	\includegraphics[scale=0.50]{imagenes/flechas_teclado.png}
	\caption{\label{fig:flechasTeclado}Teclas de movimiento del personaje}
\end{figure}

El ratón permite que el jugador apunte y dispare los proyectiles a sus enemigos con el click izquierdo.

\begin{figure}[H]
	\centering
	\includegraphics[scale=0.50]{imagenes/raton.png}
	\caption{\label{fig:raton}Raton y click izquierdo}
\end{figure}

La tecla escape permite al jugador acceder al menú de pausa
\begin{figure}[H]
	\centering
	\includegraphics[scale=0.50]{imagenes/escape.png}
	\caption{\label{fig:escape}Tecla escape}
\end{figure}

Para efectuar un salto con el personaje se utiliza la tecla espacio del teclado.

Para efectuar el cambio del arma del personaje se utilizan las teclas 1 y 2 del teclado o haciendo un giro de la rueda del ratón.

\section{Ejecución del juego}
Para abrir el juego basta con ejecutar ``ISS\_Manticore.exe''.

\section{Navegación del Menú Principal}
Al iniciarse el juego aparece el menú de inicio con las siguientes opciones: 

\begin{figure}[H]
	\centering
	\includegraphics[scale=0.40]{imagenes/MenuPrincipal3D1.png}
	\caption{\label{fig:EjemploMenuPrincipal3D}Ejemplo del menú principal}
\end{figure}

Para desplazarse entre las distintas opciones se utilizan las flechas o las teclas WASD del teclado o el ratón.

Para sprintar con el personaje, mantener presionada la tecla \textbf{shift}.

Para seleccionar la opción deseada basta pulsar la tecla enter del teclado o efectuar un click con el botón izquierdo del ratón.

A continuación, se detallan cada una de ellas: 
\begin{itemize}
	\item \textbf{Nueva Partida:} Lleva al jugador al nivel 1 como se muestra en la Figura \ref{fig:Nivel13D}.	
	\item \textbf{Instrucciones:} Activa una segunda ventana en el menú con una explicación delos controles del juego como se muestra en la Figura \ref{fig:MenuPrincipalInstrucciones}.
	\item \textbf{Créditos:} Muestra la lista de nombres de los desarrolladores y del product owner (Figura \ref{fig:MenuPrincipalCreditos}).  
	\item \textbf{Salir:} Cierra la ventana del juego. 
\end{itemize}

Cuando da comienzo la historia van apareciendo los diferentes niveles que el jugador debe superar, si este llega al final aparecerá un indicativo demostrando que ha finalizado el juego. Si por el contrario el jugador pierde todas sus vidas por el camino aparecerá otro indicativo manifestando la derrota. 	
	%\input{capitulos/Tema_6}	
	
	%\input{capitulos/5_conclusiones}
	%\includepdf[pages={57-97}]{archivos/plantilla-tfg-fium.pdf}
	%\input{capitulos/_introduccion}		% Plantilla: Se muestran contenidos
	%\input{capitulos/_marcoteorico}		% Plantilla: Se muestran listas
	%\input{capitulos/_objetivos}		% Plantilla: Se muestran tablas
	%\input{capitulos/_metodologia}		% Plantilla: Se muestran figuras
	%%%%%%%%%%%%%%%%%%%%%%%%%%%%%%%%%%%%%%%%%%%%%%%%%%%%%%%%%%%%%%%%%%%%%%%%%
% Plantilla TFG/TFM
% Escuela Politécnica Superior de la Universidad de Alicante
% Realizado por: Jose Manuel Requena Plens
% Contacto: info@jmrplens.com / Telegram:@jmrplens
%%%%%%%%%%%%%%%%%%%%%%%%%%%%%%%%%%%%%%%%%%%%%%%%%%%%%%%%%%%%%%%%%%%%%%%%

\chapter{Desarrollo (Con ejemplos de código)}
\label{desarrollo}

\section{Inserción de código}
A veces tendrás que insertar algún pedazo de código fuente para explicar algo relacionado con él. No sustituyas explicaciones con códigos enormes. Si pones algo de código en tu TFG que sea para demostrar algo o explicar alguna solución.

\LaTeX~te ayuda a escribir código de manera que su presentación tenga las marcas y tabulaciones propias de este tipo de texto. Para ello, debes poner el código que escribas DENTRO de un entorno  que se llama ``listings''.  La plantilla ya tiene una serie de instrucciones para incluir el paquete ``listings'' y añadirle algunos modificadores por lo que no tienes que incluirlo tú. Simplemente, mete tu código en el entorno ``lstlisting'' y ya está. Puedes indicar el lenguaje en el que está escrito el código y así \LaTeX~lo mostrará mejor. 
\\
\par En el archivo \textit{estiloscodigoprogramacion.tex} están definidos algunos lenguajes para mostrarlos con un diseño concreto, se pueden modificar para cambiar el coloreado del código, qué términos se ponen en negrita, etc.
Si se quiere profundizar más en la función ``listings'' se puede consultar su manual en \url{http://osl.ugr.es/CTAN/macros/latex/contrib/listings/listings.pdf}, aunque hay mucha información en foros y blog's que es más fácil de comprender.

\par Veamos un ejemplo en la figura \ref{C_code}:

\begin{lstlisting}[style=Latex-color]
\begin{lstlisting}[style=C, caption={ejemplo código C},label=C_code]
	#include <stdio.h>
	int main(int argc, char* argv[]) {
  	puts("Hola mundo!");
	}
\ end{lstlisting}	
\end{lstlisting}

El resultado será:
\begin{lstlisting}[style=C, caption={ejemplo código C},label=C_code]
#include <stdio.h>
// Comentario
int main(int argc, char* argv[]) {
  puts("Hola mundo!");
}
\end{lstlisting}
\vspace{1em}
\noindent\hrule
\vspace{1em}
Si lo quieres en color, está definido el estilo C-color en el archivo \textit{estiloscodigoprogramacion.tex}, con algunos parámetros para mejorar la visualización:
\begin{lstlisting}[style=Latex-color]
\begin{lstlisting}[style=C-color, caption={ejemplo código C en color},label=C_code-color]
#include <stdio.h>
// Comentario
int main(int argc, char* argv[]) {
  puts("Hola mundo!");
}
\ end{lstlisting}
\end{lstlisting}
\begin{lstlisting}[style=C-color, caption={ejemplo código C en color},label=C_code-color]
	#include <stdio.h>
	// Comentario
	int main(int argc, char* argv[]) {
  	puts("Hola mundo!");
	}
\end{lstlisting}
\vspace{1em}
\noindent\hrule
\vspace{1em}
Por supuesto, puedes mejorar esta presentación utilizando más modificadores. En la sección \ref{usos} se indican algunos detalles.

Otro ejemplo, ahora para mostrar código PHP, sería escribir en tu fichero \LaTeX~lo siguiente:
\begin{lstlisting}[style=Latex-color,numbers=none]
 \begin{lstlisting}[style=PHP, caption={ejemplo código PHP},label=PHP_code]
 /* 
Ejemplo de código en PHP para escribir tu primer programa en este lenguaje
Copia este código en tu ordenador y ejecútalo
*/
<html>
  <head>
    <title>Prueba de PHP</title>
  </head>
  <body>
    <?php echo '<p>Hola Mundo</p>'; ?> //esto lo escribe TODO el mundo
  </body>
</html>
 \ end{lstlisting}
\end{lstlisting}
 
 y el resultado es el siguiente:
 
 \begin{lstlisting}[style=PHP, caption={ejemplo código PHP},label=PHP_code,firstnumber=100]
/* 
Ejemplo de código en PHP para escribir tu primer programa en este lenguaje. Copia este código en tu ordenador y ejecútalo
*/
 <html>
  <head>
    <title>Prueba de PHP</title>
  </head>
  <body>
    <?php echo '<p>Hola Mundo</p>'; ?> //esto lo escribe TODO el mundo
  </body>
</html>
 \end{lstlisting}
 \vspace{1em}
\noindent\hrule
\vspace{1em}
 O también en color: 
 \begin{lstlisting}[style=PHP-color, caption={ejemplo código PHP},label=PHP_code2]
/* 
Ejemplo de código en PHP para escribir tu primer programa en este lenguaje. Copia este código en tu ordenador y ejecútalo
*/
 <html>
  <head>
    <title>Prueba de PHP</title>
  </head>
  <body>
    <?php echo '<p>Hola Mundo</p>'; ?> //esto lo escribe TODO el mundo
  </body>
</html>
 \end{lstlisting}
 
 Observa cómo \LaTeX~ha puesto los comentarios en gris y ajustado el código para que se muestre más claro.
\vspace{1em}
\noindent\hrule
\vspace{1em}
 A continuación se muestran otros ejemplos:
 \begin{lstlisting}[style=Matlab-color, caption={ejemplo código Matlab en color},label=Matlab_code]
%% Code sections are highlighted.
% System command are supported...
!touch testFile.txt
A = [1, 2, 3;... %... as is line continuation.
     4, 5, 6];
fid = fopen('testFile.text', 'w');
for k=1:10
  fprintf(fid, '%6.2f \n', k)
end
x=1; %% this is just a comment, not the start of a section
% Context-sensitive keywords get highlighted correctly...
p = properties(person); %(here, properties is a function)
x = linspace(0,1,101);
y = x(end:-1:1);
% ... even in nonsensical code.
]end()()(((end while {    end    )end ))))end (end
%{
    block comments are supported
%} even
runaway block comments are
\end{lstlisting}

\begin{lstlisting}[style=Matlab, caption={ejemplo código Matlab en blanco y negro},label=Matlab_codebn]
%% Code sections are highlighted.
% System command are supported...
!touch testFile.txt
A = [1, 2, 3;... %... as is line continuation.
     4, 5, 6];
fid = fopen('testFile.text', 'w');
for k=1:10
  fprintf(fid, '%6.2f \n', k)
end
x=1; %% this is just a comment, not the start of a section
% Context-sensitive keywords get highlighted correctly...
p = properties(person); %(here, properties is a function)
x = linspace(0,1,101);
y = x(end:-1:1);
% ... even in nonsensical code.
]end()()(((end while {    end    )end ))))end (end
%{
    block comments are supported
%} even
runaway block comments are
\end{lstlisting}
\newpage
\begin{lstlisting}[style=Python-color, caption={ejemplo código Python en color}]
class Example (object):
    def __init__ (self, account, password):
        """e.g. account  = 'bob@example.com/test'
                password = 'bigbob'
        """

        reg = telepathy.client.ManagerRegistry()
        reg.LoadManagers()

        # get the gabble Connection Manager
        self.cm = cm = reg.GetManager('gabble')

        # get the parameters required to make a Jabber connection
        # begin ex.basics.dbus.language-bindings.python.methods.call
        cm[CONNECTION_MANAGER].RequestConnection('jabber',
            {
                'account':  account,
                'password': password,
            },
            reply_handler = self.request_connection_cb,
            error_handler = self.error_cb)
        # end ex.basics.dbus.language-bindings.python.methods.call
\end{lstlisting}

\begin{lstlisting}[style=Python, caption={ejemplo código Python en blanco y negro}]
class Example (object):
    def __init__ (self, account, password):
        """e.g. account  = 'bob@example.com/test'
                password = 'bigbob'
        """

        reg = telepathy.client.ManagerRegistry()
        reg.LoadManagers()

        # get the gabble Connection Manager
        self.cm = cm = reg.GetManager('gabble')

        # get the parameters required to make a Jabber connection
        # begin ex.basics.dbus.language-bindings.python.methods.call
        cm[CONNECTION_MANAGER].RequestConnection('jabber',
            {
                'account':  account,
                'password': password,
            },
            reply_handler = self.request_connection_cb,
            error_handler = self.error_cb)
        # end ex.basics.dbus.language-bindings.python.methods.call
\end{lstlisting}

\section{Usos y personalización}
\label{usos}

El texto que acompaña al código puedes incluirlo o no, también puedes decidir si el texto va numerado o no. A continuación se muestra como:
\begin{lstlisting}[style=Latex-color]
	% Con esta línea el código no tendrá título
	\begin{lstlisting}[style=Python]
		micodigo
	\ end{lstlisting}
\end{lstlisting}

\begin{lstlisting}[style=Python]
	micodigo
\end{lstlisting}
\vspace{1em}
\noindent\hrule
\vspace{1em}
\begin{lstlisting}[style=Latex-color]
	% Con esta línea el código tendrá el título abajo
	\begin{lstlisting}[style=Python, caption={Ejemplo de título abajo},captionpos=b]
		micodigo
	\ end{lstlisting}
\end{lstlisting}

\begin{lstlisting}[style=Python,caption={Ejemplo de título abajo},captionpos=b]
	micodigo
\end{lstlisting}
\vspace{1em}
\noindent\hrule
\vspace{1em}
\begin{lstlisting}[style=Latex-color]
	% Con esta línea el código tendrá título no numerado
	\begin{lstlisting}[style=Python, title={Ejemplo de título no numerado}]
		micodigo
	\ end{lstlisting}
\end{lstlisting}

\begin{lstlisting}[style=Python,title={Ejemplo de título no numerado}]
	micodigo
\end{lstlisting}
\vspace{1em}
\noindent\hrule
\vspace{1em}
\begin{lstlisting}[style=Latex-color]
	% Con esta línea el código no tendrá las líneas numeradas
\begin{lstlisting}[style=Python,numbers=none, title={Ejemplo de código sin número de líneas}]
	micodigo
	sin
	número
	de
	líneas
\ end{lstlisting}
\end{lstlisting}

\begin{lstlisting}[style=Python,numbers=none,title={Ejemplo de código sin número de líneas}]
		micodigo
		sin
		número
		de
		líneas
\end{lstlisting}

\section{Importar archivos fuente}

Existe la posibilidad de importar un archivo de código en lugar de copiar su contenido y pegarlo en \LaTeX.

Para realizarlo debes escribir:

\begin{lstlisting}[style=Latex-color]
\lstinputlisting[style=C++-color,caption={Archivo C++ importado}]{archivos/ejemplos/holamundo.cpp}	
\end{lstlisting}

Y se importará con el formato establecido entre los '[ ]':
\newpage
\lstinputlisting[style=C++-color,caption={Archivo C++ importado}]{archivos/ejemplos/holamundo.cpp}

A continuación se muestran otros ejemplos

\begin{lstlisting}[style=Latex-color]
\lstinputlisting[style=Python-color,caption={Archivo Py importado},label=importado_py]{archivos/ejemplos/holamundo.py}	
\end{lstlisting}

\lstinputlisting[style=Python-color,caption={Archivo Py importado},label=importado_py2]{archivos/ejemplos/holamundo.py}	

\begin{lstlisting}[style=Latex-color]
\lstinputlisting[style=Matlab-color,caption={Archivo Matlab importado},label=importado_m]{archivos/ejemplos/holamundo.m}	
\end{lstlisting}

\lstinputlisting[style=Matlab-color,caption={Archivo Matlab importado},label=importado_m]{archivos/ejemplos/holamundo.m}

Ejemplo de una tablas:

\begin{table}[H]\centering
	\scalebox{0.94}{
	\setlength{\extrarowheight}{3pt}
		\begin{tabular}{rm{1.85cm}m{5cm}m{4cm}m{3.6cm}}
		\hline
			  &       & 3 Puntos & 2 Puntos & 1 Punto \\
		\hline \hline
		\multirow{12}{*}{\rotatebox[origin=rB]{90}{Ejemplo de Multifila}}    & Ortografía & Impecable & Hasta 4 errores & Más de 4 errores. \\ \cline{3-5}
			& Claridad & Todo el documento es lógico. & Una sección requiere mejor redacción. & Más de una sección requiere mejor redacción. \\ \cline{3-5}
			& Extensión & La práctica comprende entre 4 y 10 hojas. (Sin apéndices.) &       & La práctica es inferior a 4 hojas o superior a 10. (Sin apéndices.) \\ \cline{3-5}
			& Completez & La práctica cubre todas las solicitudes de trabajo. & Falta una solicitud de trabajo. & Falta más de una solicitud de trabajo. \\ \cline{3-5}
			& Gráficas & Gráficas con curvas identificadas, ejes y leyendas explicativas con unidades, y pie de figura. & Una gráfica está mal presentada. & Más de una gráfica está mal presentada. \\ \cline{3-5}
			& Diagramas/ Fotos & Claros. Ayudan a entender lo escrito. & Son poco claros. & Sólo las incluye. \\ \cline{3-5}
		7)    & Título & Corto, descriptivo y acertado. & Necesita cambios ligeros. & Requiere cambios mayores. \\ \cline{3-5}
		8)    & Resumen & Cubre brevemente por completo el trabajo realizado. & Puede redactarse mejor. & Deja puntos clave sin describir. \\ \cline{3-5}
		9)    & Introducción & Introduce clara y brevemente el trabajo realizado. & Introduce al trabajo realizado pero divaga en otros temas. & Requiere cambios mayores. \\ \cline{3-5}
		10)   & Metodología & Describe claramente lo realizado, materiales, equipo y disposición. Explica los cuidados tomados al realizar medidas y toma de datos. & Describe lo realizado en la practica y explica la realización de las medidas y la toma de datos. & Describe materiales y equipo empleado. \\ \cline{3-5}
		11)   & Resultados y Discusión & Resultados claros. Amplia explicación del significado e implicación física. Además, se comparan con otras fuentes de investigación. & Resultados con explicación de su significado físico. & Requiere mejoras en redacción y presentación. \\ \cline{3-5}
		12)   & Conclusión & Breve y cubre por completo lo realizado por el estudiante, sin divagar. & No cubre por completo lo realizado por el estudiante. & Requiere cambios mayores. \\ \cline{3-5}
		13)   & Referencias & Al menos 5 referencias, correctamente redactadas y citadas en el reporte. & Al menos 3 referencias, correctamente redactadas y citadas en el reporte.  & Al menos 1 referencia, correctamente citada.  \\ \cline{3-5}
		\hline
		\end{tabular}
	}
	\label{tab:Rubrica}
\end{table}

Ejemplo II

\begin{table}[H]\centering
  \begin{tabularx}{\textwidth}{XXXm{6.0cm}}\toprule
    Nombre & GPU  & Licencia & Descripción		\\		\midrule\midrule
    Abalone & Sí & Libre & Pliegue de proteínas y simulaciones de biomoléculas. \\
    AMBER & Sí & De Pago & Familia de campos de fuerza para dinámica molecular. \\
    CHARMM & No & Comercial & Conjunto de campos de fuerza y simulaciones de biomoléculas. \\
    DL\_POLY & Sí & Libre & Dinámica molecular en paralelo para uso general. \\
    fold.it & No & Libre & Predicción de estructura. Plegamiento molecular. \\
    GROMMACS & Sí & Libre & Simulaciones de Alto Rendimiento. \\ 
    LAMMPS & Sí & Libre & Potenciales sistemas blandos, granulares, y de estado sólido. \\
    NAMD & Sí & Libre & Computación en paralelo para dinámica molecular. \\
    TeraChem & Sí & Privado & Ab initio de alto rendimiento, optimizada para CUDA. \\
    TINKER & No & Libre & Herramientas de software para el diseño molecular. \\		\bottomrule
  \end{tabularx}
  \caption{Algunos programas de simulación en dinámica molecular.}\label{tab:ProgramasDM}
\end{table}EndFragment

Ejemplo III

\begin{table}[htbp]\centering
  \begin{tabular}{|r|c|c|c|c|c|c|c|c|c|c|c|}\hline
      & \multicolumn{5}{c|}{2017}      & \multicolumn{6}{c|}{2018} \\ \cline{2-12}
      & Ago & Sep & Oct & Nov & Dic & Ene & Feb & Mar & Abr & May  & Jun \\ \hline
    \makecell[r]{Revisión \\ Bibliográfica} & \cclg & \cclg & \cclg & \cclg &	&	&	&	&	&	&  \\ \hline
    \makecell[r]{Escritura \\ del Protocolo} &	&	& \cclg &	&	&	&	&	&	&	&  \\ \hline
    \makecell[r]{Tema \\ A} & &	&	& \cclg & \cclg & \cclg &	&	&	&	&  \\ \hline
    \makecell[r]{Tema \\ B} & &	&	&	& \cclg & \cclg & \cclg &	&	&	&  \\ \hline
    \makecell[r]{Tema \\ C} & &	&	&	& \cclg & \cclg & \cclg &	&	&	&  \\ \hline
    \makecell[r]{Tema \\ D} & &	&	&	&	& \cclg & \cclg & \cclg &	&	&  \\ \hline
    \makecell[r]{Tema \\ E} & &	&	&	&	&	& \cclg & \cclg & \cclg &	&  \\ \hline
    Escritura de Tesis & &	&	&	&	&	& \cclg & \cclg & \cclg & \cclg &  \\ \hline
    Escritura de Artículo & &	&	&	&	&	&	&	& \cclg & \cclg & \cclg \\ \hline
    Defensa & &	&	&	&	&	&	&	&	&	& \cclg \\ \hline
  \end{tabular}
  \caption{Cronograma de Actividades.}
  \label{tab:CronoAct}
\end{table}
		% Plantilla: Se muestran listados
	%\input{capitulos/_resultados}		% Plantilla: Se muestran gráficas
	%\input{capitulos/_conclusiones}		% Plantilla: Se muestran matemáticas
	
	%%%%
	% CONTENIDO. BIBLIOGRAFÍA.
	%%%%
	\nocite{*} %incluye TODOS los documentos de la base de datos bibliográfica sean o no citados en el texto
	\bibliographystyle{unsrtnat}
	\bibliography{bibliografia/bibliografia} % Archivo que contiene la bibliografía
	
	
	%%%%
	% CONTENIDO. LISTA DE ACRÓNIMOS. Comenta las líneas si no lo deseas incluir.
	%%%%
	% Incluye el listado de acrónimos utilizados en el trabajo. 
	%\printglossary[style=modsuper,type=\acronymtype,title={Lista de Acrónimos y Abreviaturas}]
	% Añade el resto de acrónimos si así se desea. Si no elimina el comando siguiente
	%\glsaddallunused
	
	%%%%
	% CONTENIDO. Anexos - Añade o elimina según tus necesidades
	%%%%
	%\appendix % Inicio de los apéndices
	%%%%%%%%%%%%%%%%%%%%%%%%%%%%%%%%%%%%%%%%%%%%%%%%%%%%%%%%%%%%%%%%%%%%%%%%%
% Plantilla TFG/TFM
% Universidad de Murcia. Facultad de Informática
% Realizado por: José Manuel Requena Plens
% Modificado: Pablo José Rocamora Zamora
% Contacto: pablojoserocamora@gmail.com
%%%%%%%%%%%%%%%%%%%%%%%%%%%%%%%%%%%%%%%%%%%%%%%%%%%%%%%%%%%%%%%%%%%%%%%%

\chapter{Anexo I}

Aquí vendría el anexo I 
	%%%%%%%%%%%%%%%%%%%%%%%%%%%%%%%%%%%%%%%%%%%%%%%%%%%%%%%%%%%%%%%%%%%%%%%%%
% Plantilla TFG/TFM
% Universidad de Murcia. Facultad de Informática
% Realizado por: José Manuel Requena Plens
% Modificado: Pablo José Rocamora Zamora
% Contacto: pablojoserocamora@gmail.com
%%%%%%%%%%%%%%%%%%%%%%%%%%%%%%%%%%%%%%%%%%%%%%%%%%%%%%%%%%%%%%%%%%%%%%%%


% Ejemplo de páginas en horizontal y vertical

\chapter{Páginas horizontales}
Aquí se muestra cómo incluir páginas en horizontal.

Esta página está en vertical\\
\clearpage % Nueva página

\begin{landscape} % Inicia modo horizontal
	

Esta página está en horizontal\\
\clearpage % Nueva página

Esta página también está en horizontal\\

\end{landscape} % Finaliza modo horizontal
\clearpage % Nueva página


Esta página está de nuevo en vertical\\




	%\input{anexos/_anexo_3}
	
\end{document}